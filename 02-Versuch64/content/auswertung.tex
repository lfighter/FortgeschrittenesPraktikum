\section{Auswertung}
\label{sec:Auswertung}
Die Graphen wurden sowohl mit Matplotlib \cite{matplotlib} als auch NumPy \cite{numpy} erstellt. Die
Fehlerrechnung wurde mithilfe von Uncertainties \cite{uncertainties} durchgeführt.
\subsection{Bestimmung des optimalen Winkels am Polarisationsfilter}
\begin{figure}
	\centering
	\includegraphics[width=\linewidth-70pt,height=\textheight-70pt,keepaspectratio]{build/Kontrast.pdf}
	\caption{Der Kontrast des Interferometer gegen den Winkel des Polarisationsfilters aufgetragen.}
\end{figure}
\begin{table}
	\centering
	\caption{Die maximale Spannung $U_\text{max}$ und die minimale Spannung $U_\text{min}$ an der Diode und der daraus berechnete Kontrast für die verschiedenen Winkel $\phi$ am Polarisationsfilter.}
	\input{build/Kontrast.tex}
\end{table}
Der Kontrast in Tabelle \ref{tab:Kontrast} wird mit Formel \eqref{eq:kont} aus den am Oszilloskop gemessenen Spannungen $U_\text{min}$ und $U_\text{max}$ berechnet. Der Fit wird mit einer nicht linearen Ausgleichsrechnung der Form $a |\sin(2(\phi+\delta\phi))|$ bestimmt. Hierbei wird die Funktion curve\_fit aus der NumPy-Bibliothek SciPy \cite{scipy} verwendet. Die Parameter berechnen sich zu
\begin{gather*}
a=\num{0.95(1)}\\
\delta\phi=\SI{3.1(4)}{\degree}.
\end{gather*}
Daraus ergibt sich der maximale Kontrast $K_\text{max}=a=\num{0.95(1)}$ bei einem Winkel am Polarisationsfilter von $\phi=\SI{45}{\degree}+\delta\phi=\SI{48.1(4)}{\degree}$.

\subsection{Bestimmung des Brechungsindexes der Glasplatten}
\begin{table}
	\centering
	\caption{Die gemessene Anzahl an Übergängen $M$ bei einer Drehung von jeweils $\SI{10}{\degree}$.}
	\input{build/Glas.tex}
\end{table}
Mit den Werten aus Tabelle \ref{tab:Glas} ergibt sich für den Mittelwert $M_\text{mittel}=\num{34.1(4)}$. Damit bestimmt sich der Wert für $n_\text{Glas}$ nach Formel \eqref{} zu
\begin{displaymath}
	n=\num{1.302(4)}.
\end{displaymath}

\subsection{Bestimmung des Brechungsindexes von Luft}

\begin{figure}
	\centering
	\includegraphics[width=\linewidth-70pt,height=\textheight-70pt,keepaspectratio]{build/Luft.pdf}
	\caption{Der jeweils aus der Anzahl an Übergängen $M$ berechnete Brechungsindex $n$ gegen den Luftdruck $p$ aufgetragen.}
	\label{fig:Luft}
\end{figure}
\begin{table}
	\centering
	\caption{Die Anzahl an gemessenen Übergängen $M$ für verschiedene Luftdrücke $p$.}
	\input{build/Luft.tex}
\end{table}
Die in Abbildung \ref{fig:Luft} berechneten Wirkungsindexe wurden mit Formel \eqref{eq:Deltaphi} berechnet. Der Fit besitzt die Form $f(x,a,b) =a\cdot x +b$ und wurde mit Hilfe einer linearen Regression bestimmt. Für die Parameter ergeben sich
\begin{gather*}
	a=\SI{5.29(3)e-7}{\per\milli\bar}\\
	b=\num{0.999996(2)}.
\end{gather*}
Durch Einsetzten der bestimmten Parameter in $\sqrt{f(x,a,b)}$ ergibt sich für einen Luftdruck von \SI{1013}{\milli\bar} und bei \SI{26.5}{\degreeCelsius} nach dem Lorenz-Lorentz Gesetz
\begin{displaymath}
 n_\text{Luft}(\SI{1013}{\milli\bar},\SI{26.5}{\degreeCelsius})=\num{1.000266(2)}.
\end{displaymath}