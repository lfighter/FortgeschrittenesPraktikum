\section{Aufbau}
Der Aufbau besteht aus einem Sagnac Interferometer in Abb. \ref{}. Die Stärke des Sagnac Interferometers ist eine naturgemäße Stabilität gegenüber äußeren Störeinflüssen wie zum Beispiel Vibrationen der Spiegel oder kleinen Luftturbolenzen. Zu Beginn wird der Strahl eines HeNe Lasers auf einen über zwei Ausrichtungsspiegel auf einen PBSC gerichtet. Der Strahl ist linear polarisiert und die Quelle ist so gedreht, dass der Vektor des elektromagnetischen Feldes in einem $\SI{45}{\degree}$ Winkel zur Vertikalen Achse steht. Der eintreffende Strahl wird durch einen Strahlteiler in zwei Strahlen, deren Polarisierungen orthogonal zueinander liegen, geteilt. Als Strahlteiler wird ein PBSC verwendet. Zusätzlich gelangt einer der Strahlen gradlinig durch den Strahlteiler, während der zweite Strahl um $\SI{90}{\degree}$ gedreht wird. Beide Strahlen gelangen danach gegenläufig durch die in Abb. \ref{} dargestellte Spiegelkonstruktion aus 3 Spiegeln, welche jeweils im $\SI{45}{\degree}$ Winkel zu den Strahlen aufgestellt wird. Bei genauer Ausrichtung sollen sich beide Strahlen vollständig überlappen, jedoch in entgegengesetzte Richtungen laufen. Am Strahlteiler laufen beide Strahlen wieder zusammen und werden über die vierte Kante wieder zu einem Strahl zusammengeführt. An diesem Punkt liegt jedoch weiterhin keine Phasendifferenz vor. 