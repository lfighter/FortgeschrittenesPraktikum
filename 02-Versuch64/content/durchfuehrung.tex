
\section{Durchführung}
\label{sec:Durchführung}
Die Durchführung besteht zum Großteil daraus, die einzelnen Spiegel des Interferometers derartig auszurichten, sodass sich beide Strahlen hinreichend genau überlappen.
Ales erstes werden der HeNe Laser, die zugehörigen Umlenkspiegel, der PBSC und der erste Spiegel $M_\text{A}$ gemäß Abb.\ref{} auf dem Konstruktionstisch montiert. Der umgelenkte Strahl kann durch eine Abschirmung blockiert werden. Anschließend werden zwei Lochblenden an die Positionen 1 und 2 in den zweiten Strahlengang gebracht und die Lenkspiegel $M_text{1}$ und $M_text{2}$ werden so justiert, sodass der Strahl beide Blenden passiert und auf $M_\text{A}$ trifft. Nun wird der Spiegel $M_\text{C}$ montiert. Hierzu werden die Lochblenden an die Positionen 8 und 9 gebracht und die Abschirmung wird vor den graden Strahlen gesetzt. 
%kein plan wie die das bei schritt 2 mit der halterung meinen, dann verdreh ich doch meinen graden strahl wieder
Zur Justierung werden die Schrauben der Strahlteilerhalterung gelockert und die Halterung wird so gedreht, dass der Strahl die Blenden passiert. Nun wird letztendlich auch der letzte Spiegel $M_\text{B}$ montiert. Dazu werden zunächst die Lochblenden an die Positionen 5 und 6 gebracht. Anschließend wird zunächst $M_\text{A}$ so justiert, sodass der Strahl die Lochblende bei Position 5 passiert. Das gleiche wird für den Spiegel $M_\text{C}$ wiederholt. Nach Entfernung der Blenden, sollte $M_\text{B}$ von zwei seperaten Strahlen getroffen werden, welche sich in einem Punkt auf der Spiegeloberfläche treffen. Die Spiegel $M_\text{A}$ und $M_\text{C}$ können nun festgestellt werden. Bevor auch $M_\text{B}$ festgestellt wird, wird vorher überprüft, dass die Strahlen ihn auch von beiden Seiten treffen. Um die Positionen der Strahlengänge zu erfassen, kann ein Stück Papier verwendet werden. $M_\text{B}$ wird nun so gedreht, sodass sich beide Strahlen überlappen. Nun kann auch $M_\text{B}$ festgestellt werden. Stelle danach die Abschirmung hinter die vierte Seite des Strahlteilers. Im Regelfall sollte der zusammengeführte Strahl sich noch nicht überlappen. Zur Feinjustierung können die Justierschrauben an $M_\text{A}$ und $M_\text{C}$ verwendet werden. Der erzeugte Strahl sollte nun einen Überlapp besitzen, jedoch werden keine Beugungsmuster zu erkennen sein. Um die Ursache zu erkennen wird ein Polaroid zwischen Strahlteiler und Beobachtungsschirm gesetzt und der Strahl wird wieder leicht aufgetrennt. Ist die Polarisierung klar geworden wird der Polaroid in einem $\SI{45}{\degree}$ Winkel befestigt, sodass beide Strahlen zu Teil hindurchgelangen. Danach werden beide Strahlen wieder zum Überlapp gebracht. Zuletzt wird versucht die Ortsfrequenz der Beugungsmuster zu minimieren.   