
\section{Durchführung}
\label{sec:Durchführung}
Die Durchführung besteht zum Großteil daraus, die einzelnen Spiegel des Interferometers geeignet auszurichten. %sodass sich beide Strahlen hinreichend genau überlappen.
Ales erstes werden der HeNe Laser, die zugehörigen Umlenkspiegel, der PBSC und der erste Spiegel $M_\text{A}$ gemäß Abb.\ref{} auf dem Konstruktionstisch montiert. Der umgelenkte Strahl kann durch eine Abschirmung blockiert werden. Anschließend werden zwei Lochblenden an die Positionen 1 und 2 in den zweiten Strahlengang gebracht und die Lenkspiegel $M_text{1}$ und $M_text{2}$ werden so justiert, sodass der Strahl beide Blenden passiert und auf $M_\text{A}$ trifft. Nun wird der Spiegel $M_\text{C}$ montiert. Hierzu werden die Lochblenden an die Positionen 8 und 9 gebracht und die Abschirmung wird vor den graden Strahlen gesetzt. 
%kein plan wie die das bei schritt 2 mit der halterung meinen, dann verdreh ich doch meinen graden strahl wieder
Zur Justierung werden die Schrauben der Strahlteilerhalterung gelockert und die Halterung wird so gedreht, dass der Strahl die Blenden passiert. Nun wird letztendlich auch der letzte Spiegel $M_\text{B}$ montiert. Dazu werden zunächst die Lochblenden an die Positionen 5 und 6 gebracht. Anschließend wird zunächst $M_\text{A}$ so justiert, sodass der Strahl die Lochblende bei Position 5 passiert. Das gleiche wird für den Spiegel $M_\text{C}$ wiederholt. Nach Entfernung der Blenden, sollte $M_\text{B}$ von zwei seperaten Strahlen getroffen werden, welche sich in einem Punkt auf der Spiegeloberfläche treffen. Die Spiegel $M_\text{A}$ und $M_\text{C}$ können nun festgestellt werden. Bevor auch $M_\text{B}$ festgestellt wird, wird vorher überprüft, dass die Strahlen ihn auch von beiden Seiten treffen. Um die Positionen der Strahlengänge zu erfassen, kann ein Stück Papier verwendet werden. $M_\text{B}$ wird nun so gedreht, sodass sich beide Strahlen überlappen. Nun kann auch $M_\text{B}$ festgestellt werden. Stelle danach die Abschirmung hinter die vierte Seite des Strahlteilers. Im Regelfall sollte der zusammengeführte Strahl sich noch nicht überlappen. Zur Feinjustierung können die Justierschrauben an $M_\text{A}$ und $M_\text{C}$ verwendet werden. Der erzeugte Strahl sollte nun einen Überlapp besitzen, jedoch werden keine Beugungsmuster zu erkennen sein. Um die Ursache zu erkennen wird ein Polaroid zwischen Strahlteiler und Beobachtungsschirm gesetzt und der Strahl wird wieder leicht aufgetrennt. Ist die Polarisierung klar geworden wird der Polaroid in einem $\SI{45}{\degree}$ Winkel befestigt, sodass beide Strahlen zu Teil hindurchgelangen. Danach werden beide Strahlen wieder zum Überlapp gebracht. Zuletzt wird versucht die Ortsfrequenz der Beugungsmuster zu minimieren. Hierzu werden die Lochblenden wieder auf die Positionen 2 und 3 gebracht und $M_\text{2}$ wird über die Laufschiene soweit verschoben, bis der Strahl nicht mehr durch die mittlere Blendeöffnung, sondern durch die Seitlichen fällt. Gegebenenfalls müssen $M_\text{1}$ und $M_\text{2}$ auch in der Höhe verstellt werden. Es ist hilfreich den umgelenkten Strahl während der Justierung zu blockieren.
%alles nach stelle 8
Nun werden die Lochblenden wieder entfernt und es sollten im Interferometer zwei räumlich getrennte Strahlen zu erkennen sein, welche das Polaroid durchlaufen und schließlich wieder auf dem Schirm überlappen. Zusätzlich wird nochmals die Polarisation beider Strahlen getestet. Um die Beugungsmuster zu optimieren kann wieder an den Daumenschrauben an $M_\text{A}$ und $M_\text{C}$ justiert werden.
%schritt 16
Nun wird die Halterung zur Bestimmung des Brechungsindex von dünnen Schreiben in die Strahlengänge der zwei parallelen Strahlen gebracht. Die integrierten Glasscheiben sind so angebracht, sodass sich beide unter einer Drehung in gegenläufige Richtungen drehen. Dies erhöht die erkennbare Phasendifferenz. Unter Rotation bilden sich nun gleichmäßig ändernde Beugungsmuster aus, welche bei einem Stopp jedoch stabil gegenüber äußeren Störungen bleiben.
%schriit 11
da nun gezeigt worden ist, dass sich Beugungsmuster mithilfe eines Polaroids erzeugen lassen, wird dieses gegen eine effizientere Kombination ausgetauscht. Dazu wird der Polaroid entfernt und ein zweiter Strahlteiler wird im $\SI{45}{\degree}$ Winkel hinter dem Ausgang des Sagnac Interferometers gemäß Abb. \ref{} montiert. Der hier grade Strahl entspricht dem Überlapp und den im Interferometer auftretenden Interferenzen. Daher sollten bei Hinzunahme einer Leinwand Beugungsmuster zu erkennen sein. Der umgelenkte Strahl wird über einen weiteren Spiegel ebenfalls auf die Leinwand gebracht. Anschließend wird das Interferometer so eingestellt, sodass an beiden Auftreffpunkten keine Ortsfrequenz mehr zu erkennen ist. Die Lichtintensitäten sollten nun zwischen beiden Punkten komplementär sein. Danach wird die Leinwand gegen zwei Photodetektoren ausgetauscht, ihre Zentren werden auf die Strahlengänge ausgerichtet. Der Gain wird bei beiden gleich gewählt. Beide Intensitäten werden anschließend über ein Zweikanaloszilloskop ausgelesen. Auch hier sollte das komplementäre Verhalten zwischen beiden Strahlen zu erkennen sein. Um eine bessere Darstellung zu erhalten wird die Differenz zwischen beiden Signalen betrachtet. Dazu wird ein Signal-Fluss-Schema dazwischengeschaltet. Es sollte zu erkennen sein, dass das Signal nur sehr kleinen Untegrund besitzt. Da im Interferometer jedoch weiterhin zwei parallele Strahlen verwendet werden, kommt es zu Anfälligkeiten gegenüber Luftturbolenzen. Daher wird zusätzlich ein Windschutz über das Interferometer gestellt. Es ist nun stabil gegenüber Luftturbolenzen und Vibrationen. Da nun alle Komponeneten justiert sind, kann die eigentliche Messung begonnen werden. 