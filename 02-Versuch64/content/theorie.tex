
\section{Theorie}
\label{sec:Theorie}
\subsection{Der Brechungsindex}
Der Brechungsindex gibt das Verhältnis der Lichtgeschwindigkeit in Materie und der Lichtgeschwindigkeit im Vakuum. Daher gilt:
\begin{equation}
    n = \frac{c_\text{Materie}}{c_\text{Vakuum}} \label{eqn:n} \text{.}
\end{equation}  
 \subsection{Die Bestimmung des Brechungsindex von Gasen}
 In der klassischen Darstellung wird Licht als Überlagerung ebener Wellen dargestellt. Wird zusätzlich linear polarisiertes Licht angenommen, kann Die Welle durch
 \begin{equation}
    f(x,t) = exp(i \frac{2 \pi}{\lambda_\text{Vakuum}} n x ) e^{(-i \omega t)} \label{eqn:ebeneWelle}
 \end{equation}
dargestellt werden. Kommt es nun zu einer Änderung des Brechungsindex über eine Länge $L$,so führt dies zu einer Phasenverschiebung der Welle. Bei einem Vakuumbrechungsindex von eins gilt:
\begin{equation}
    \Delta \varphi = \frac{2 \pi}{\lambda_\text{Vakuum}} (n-1) L \text{.} \label{eqn:Deltaphi}
\end{equation}
Der Brechungsindex eines Gases beziehungsweise einer Flüssigkeit in Abhängigkeit des Druckes, der Temperatur und der Polarisierbarkeit lässt sich gut durch das Lorenz-Lorentz Gesetz beschreiben. Es ist zudem möglich die Eigenschaften von Gas oder Flüssigkeitkeitsgemischen zu bestimmen, indem die Eigenschaften der einzelnen Komponenten bestimmt werden.


\subsection{Die Bestimmung des Brechungsindex von lichtdurchlässigen Festkörpern}


\subsection{Die Polarisation des verwendeten Lichtes}

\subsection{Der PBSC }