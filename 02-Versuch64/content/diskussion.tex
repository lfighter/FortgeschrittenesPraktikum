
\section{Diskussion}
\label{sec:Diskussion}
Der in Abschnitt \ref{ssec:polar} bestimmte Wert für den optimalen Winkel des Polarisationsfilters weicht mit $\phi_\text{max}=\SI{48.1(4)}{\degree}$ um $\SI{3.1(4)}{\degree}$ vom theoretischen Winkel von $\SI{45}{\degree}$ nach oben ab. Dies könnte zu einem daran liegen, dass der Winkel der am Polarisationsfilter eingestellt ist nicht dem Polarisationswinkel des Lichtes hinter dem Filter entspricht. Auch könnte der dahinter liegende PBSC leicht gedreht sein oder einen der zwei Polarisationsrichtungen bevorzugt durchlassen.
Der im Abschnitt \ref{ssec:glas} bestimmte Wert für den Brechungsindex vom Glas $n_\text{Glas}=\num{1.302(4)}$ liegt ?????. Mögliche Ursachen für eine Abweichung vom tatsächlichen Wert sind zu einem, dass der Winkel zu schnell oder zu ungleichmäßig verändert wurde, so dass Maxima nicht oder mehrfach gezählt wurden. Auch dass die Scheiben für einen Winkel von $\theta=\SI{0}{\degree}$ nicht ganz senkrecht zu den Strahlen sind könnte Abweichungen verursachen.
Die Messwerte in Abbildung \ref{fig:Luft} in Abschnitt \ref{ssec:luft} liegen gut auf der bestimmten Ausgleichsgerade. Sie besitzen eine Abweichung zu dieser von $\sigma_y=\num{0}$. Der im selben Kapitel bestimmte Wert für den Brechungsindex von Luft $n_\text{Luft}=\num{1.000266(2)}$ bei \SI{1013}{\milli\bar} und \SI{26.5}{\degreeCelsius} weicht um \num{1(2)e-6} vom Literaturwert $n_\text{lit}=\num{1.000265205(32)}$ \cite{nist} nach oben ab. Dies kann durch die geringe Auflösung für die Anzahl an übergangenen Maxima begründet werden.



