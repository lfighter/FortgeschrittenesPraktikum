\section{Aufbau}
\label{sec:Aufbau}

%bild vom aufbau einfügen
Die im Detektor entstandenen Ladungsimpulse müssen nun noch verarbeitet werden. Hierzu gelangen die Impulse zunächst in einen Vorverstärker. In diesem wird eine Spannung erzeugt, welche proportional zum über die Zeit integrierten Strom ist. Damit die Spannung nach einem registrierten Impuls wieder auf Null zurückfällt und sich nicht über mehrere Impulse hin addiert muss sie nach dem Ende des Impulses wieder abfließen. Da ein paralell geschalteter Wiederstand jedoch zu verstärktem Rauschen führen würde wird eine optoelektronische Schaltung verwendet. Diese lässt den strom abfließen, sobald die Spannung eine obere Schwelle überschreitet. Um Rauscheffekte zu minimieren werden sowohl Detektor als auch Vorverstärker mit flüssigem Stickstoff auf $\SI{77}{\kelvin}$ gekühlt. Die Hochspannungsversorgung ist zudem an einen Temperaturwächter gekoppelt, da die Spannung bei höheren Temperaturen das Detektormaterial beschädigen würde. Da ein plötzlicher Spannungsabfall aufgrund von höheren Temperaturen jedoch den Vorverstärker beschädigen würde ist zudem ein RC-Glied mit großer Zeitkonstante integriert. Zusätzlich ist der Detektor von einer Schutzhaube aus Aluminium umgeben, welche dafür sorgt, dass Gammaquanten mit Energien unterhalb von $\SI{40}{\kilo\electronvolt}$ nicht registriert werden können. Anschließend gelangt die Spannung über eine kapazitative Signalleitung mit niedriger Impedanz in den Hauptverstärker. Dieser verstärkt die Signale auf einen Spannungsbereich von $0 - \SI{10}{\volt}$. Es ist darauf zu achten, dass die Verstärkung zeitlich konstant bleibt und die Bandbreite des Verstärkers geeignet gewählt wird. Letzteres reduziert zusätzliches rauschen. Danach werden die Impulse in einem Analog zu Digital Konverter, kurz ADC, in digitale Signale umgewandelt. In diesem wird mithilfe der eingehenden Spannung ein Kondensator aufgeladen. Danach entlädt sich dieser wieder über eine Konstantstromquelle und die dafür benötigte Zeit wird mithilfe eines Quarzosszillators gemessen. Wenn am Kondensator wieder eine Spannung von Null anliegt wird die Zählung beendet und das ergebnis an einen Binärzähler weitergeleitet. Damit während des Entladevorgangs kein weiterer Impuls in den ADC gelangt wird dessen Eingang über eine integrierte Steuereinheit für die benötigte Zeit gesperrt. Diese liegt bei etwa $\SI{40}{\micro\second}$. Sie können durch die Unterscheidung zwischen der Messzeit, ohne die Totzeiten und der Echtzeit berücksichtigt werden. Zuletzt gelangen die Signale in einen Vielkanalanalysator. In diesem werden die Signale anhand ihrer Höhe, welche proportional zur Energie ist, in verschiedene Kanäle sortiert und es wird ihre jeweilige Anzahl gezählt. Der verwendete Detektor besitzt 8192 Kanäle. Nach der Messung erhält man das Spektrum des Strahlers.