
\section{Diskussion}
\label{sec:Diskussion}
Die Auswertung hat einige Ergebnisse erbracht, welch nun noch zu diskutieren sind. Zuerst wurden die einzelnen Kenngrößen des Detektors mithilfe einer kalibrierten $^{152}$Eu Probe bestimmt. Die Kalibrierung des bestimmten Umrechnungsfaktor von Kanälen zu Energien zeigt nur geringe Unsicherheiten, was auf die geringen Halbwertsbreiten und die daraus resultierende, hohe Auflösung des Reinst-Germaniumdetektors schließen lässt. Die großen Unsicherheiten der Effizienzmessung lassen sich auf die Form des verwendeten Fittes zurückführen. Aufgrund des Potenzansatzes und den Unsicherheiten auf den nur wenigen Messdaten ergeben sich viele Parameterkombinationen, welche sich zwar grundlegend unterscheiden, jedoch fast dieselbe Kurve erzeugen. Um dieses Problem zu minimieren war darauf zu achten einen Strahler mit möglichst reichhaltigem Spektrum zu Kalibrierung zu verwenden. Die Fehler, welche nach der Poisson-Statistik eigentlich auf die Höhe der einzelnen Kanäle genommen werden würden, wurden jedoch vernachlässigt, da sie klein im Vergleich zur realen Kanalbesetzung sind. Um die Messung nicht durch nur teilweise registrierte Spektren zu verfälschen wurden zusätzlich nur Werte oberhalb von $\SI{150}{\kilo\electronvolt}$ verwendet. Die Untersuchung des Spektrums von $^{137}$Cs ergab die erwarteten Werte und die experimentell bestimmten Werte liegen bei allen Untersuchten Größen in der Nähe der theoretisch bestimmten Werte. Es ist jedoch zu beachten, dass für die Energie am Vollenergiepeak immer der selbst bestimmte Wert verwendet wurde. Beim Vergleich des bestimmten Verhältnis von Zehntel und Halbwertsbreite mit dem Verhältnis, welches eine Gaußkurve besitzt, ist zu erkennen, dass diese nah beieinander liegen. Daher können die vermessenen Peaks gut durch Gaußkurven angenähert werden. Allerdings ist zu erkennen, dass die experimentelle Halbwertbreite doppelt so groß ausfällt wie die theoretisch bestimmte. Dies ist auf die in der Theorie beschriebenen Rauscheffekte zurückzuführen. Da der Detektor jedoch heruntergekühlt wird, kann Rauschen infolge von Leckströmen vernachlässigt werden, weshalb das Rauschen entweder durch den Verstärker oder durch Inhomogenitäten im Feld verursacht wird. Anhand der bestimmten Verhältnisse zwischen experimentellem und theoretischem Auftreten von Photo- und Comptoneffekt ist zudem zu erkennen, dass sich die Wahrscheinlichkeit eines Photoeffekts bei höheren Energien nicht schnell verringert wie theoretisch angenommen. 
Die Spektren der beiden untersuchten unbekannten Proben stimmen gut mit den Spektren der zugeordneten Elemente überein.