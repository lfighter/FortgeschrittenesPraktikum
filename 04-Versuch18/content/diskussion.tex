
\section{Diskussion}
\label{sec:Diskussion}
Die Auswertung hat einige Ergebnisse erbracht, welche nun noch zu diskutieren sind. Zuerst wurden die einzelnen Kenngrößen des Detektors mithilfe einer kalibrierten $^{152}$Eu Probe bestimmt. Der bei der Kalibrierung bestimmte, lineare Zusammenhang zwischen Kanälen und Energien zeigt nur geringe INPROZEZNT Unsicherheiten. Da die Standardabweichungen der bestimmten Peaks in Tabelle \ref{tab:datsigmatab} mit maximal $\num{6.6(8)}$ bei 8192 Kanälen nur gering ausfallen, folgt aus der Gaußförmigkeit des Fittes, dass die Peaks nur geringe Halbwertsbreiten besitzen. Dies lässt auf die hohe Auflösung des Reinst-Germaniumdetektors schließen.
%satz grammatikalisch überprüfen

Um Unsicherheiten zu minimieren war darauf zu achten einen Strahler mit möglichst reichhaltigem Spektrum zu Kalibrierung zu verwenden. Die Unsicherheiten, welche nach der Poisson-Statistik eigentlich auf die Höhe der einzelnen Kanäle genommen werden würden, wurden jedoch vernachlässigt, da sie klein im Vergleich zur realen Kanalbesetzung sind.

Die Untersuchung des Spektrums von $^{137}$Cs ergab die erwarteten Werte und die experimentell bestimmten Werte liegen bei allen untersuchten Größen in der Nähe der theoretisch bestimmten Werte. Es ist jedoch zu beachten, dass für die Energie am Vollenergiepeak immer der selbst bestimmte Wert verwendet wurde. Beim Vergleich des bestimmten Verhältnis von Zehntel und Halbwertsbreite mit dem Verhältnis, welches eine Gaußkurve besitzt, ist zu erkennen, dass diese nah beieinander liegen. Daher können die vermessenen Peaks gut durch Gaußkurven angenähert werden. Allerdings ist zu erkennen, dass die experimentelle Halbwertbreite doppelt so groß ausfällt wie die theoretisch bestimmte. Die Ursache hierfür sind die in Kapitel \ref{subsec:eighat} beschriebenen Halbwertsbreiten aufgrund von Rauscheffekten $H_\text{R}$, $H_\text{I}$ und $H_\text{E}$, aus welchen sich die Energieauflösung $\H_\text{Ges}$ gemäß \eqref{eq:Hges} ergibt. Da der Detektor jedoch heruntergekühlt wird, ist die Halbwertsbreite des Rauschen infolge von Leckströmen $H_\text{R}$ vernachlässigbar klein. Daher werden die zusätzlichen Halbwertsbreiten des Rauschens entweder durch den Verstärker ($H_\text{E}$) oder durch Inhomogenitäten im Feld ($H_\text{I}$) verursacht. Anhand der bestimmten Verhältnisse zwischen experimentellem und theoretischem Auftreten von Photo- und Comptoneffekt ist zudem zu erkennen, dass sich die Wahrscheinlichkeit eines Photoeffekts bei höheren Energien nicht schnell verringert wie theoretisch angenommen. Dies lässt sich durch Fall eines Comptoneffektes gefolgt von einem Photoeffekt erklären. Bei diesem wird die Energie des Gammaquants zunächst aufgrund des Comptoneffektes verringert, weswegen die Wahrscheinlichkeit eines Photoeffektes gemäß Abb. \ref{fig:effekt} steigt.
Die Spektren der beiden untersuchten unbekannten Proben stimmen gut mit den Spektren der zugeordneten Elemente überein.