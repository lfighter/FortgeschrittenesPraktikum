
\section{Durchführung}
\label{sec:Durchführung}
Zunächst wird der Abstand der Probe zum Detektor mithilfe eines Stabes derselben Länge ermittelt. Anschließend wird ein kalibrierter $^{152}$Eu Strahler verwendet, um damit eine Energiekalibrierung der Apparatur und eine Messung der Vollenergienachweiswahrscheinlichkeit des Detektors durchzuführen. Aus den Daten dieser Messung werden später in der Auswertung die Anzahl und die Lage der Peaks bestimmt. Danach wird das Spektrum eines $^{137}$Cs-Strahlers aufgenommen. An dessen Spektrum wird der Vollenergiepeak und das Compton-Kontinuum genauer untersucht. Auch die Lage des Rückstreupeaks wird bestimmt. Im Anschluss wird das Spektrum eines $^{125}$Sb oder eines $^{133}$Ba Quelle vermessen, um daraus die Aktivität der Probe zu bestimmen. Zuletzt wird noch das Spektrum eines unbekannten Strahlers aufgenommen, um damit das verwendete Element zu bestimmen. Zusätzlich soll auch die Aktivität der Probe des unbekannten Strahlers ermittelt werden.