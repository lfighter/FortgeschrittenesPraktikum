
\section{Durchführung}
\label{sec:Durchführung}
Zunächst wird der Abstand der Probe zum Detektor mithilfe eines Stabes derselben Länge ermittelt. Anschließend wird ein kalibrierter $EU^{152}$ Strahler verwendet um nachher eine Energiekalibrierung der Apparatur und eine Messung der Effizienz des Detektors durchzuführen. Aus den Daten dieser Messung werden später in der Auswertung die Anzahl und die Lage der Peaks ermittelt. Danach wird das Spektrum eines $Cs^{137}$ Strahlers aufgenommen. Die Daten werden analog zum $Eu^{152}$ Strahler ausgewertet. Danach wird das Spektrum eines $Sb^{125}$ oder eines $Ba^{133}$ Quelle vermessen um daraus die Aktivität der Probe zu bestimmen. Zuletzt wird das Spektrum eines unbekannten Strahlers aufgenommen um damit das verwendete Element zu bestimmen.