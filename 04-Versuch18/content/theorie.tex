
\section{Theorie}
\label{sec:Theorie}
\subsection{Absorption von Gammastrahlung durch Materie}
%erster satz gefällt mir vom anfang nich, besseren suchen
Wird feste Materie mit Gammastrahlung bestrahlt  kommt es zu Wechselwirkungen der einzelnen Gammaquanten mit den Atomen und Elektronen des Festkörpers, wodurch die Intensität des Gammastrahles entlang seines Weges durch den Festkörper reduziert wird. Die drei dominierenden Effekte sind dabei Photoeffekt, Comptoneffekt und Paarbildung. Der Beitrag der jeweiligen Wechselwirkung an der gesamten Intensitätsabnahme wird durch den zugehörigen Wirkungsquerschnitt beschrieben. Anschaulich lässt sich dieser wie die Größe einzelner Zielscheiben deuten, welche den Einfangsradius der Atome und Elektronen beschreiben. Die Größe der Zielscheiben, also das $\sigma$ ist so gewählt, sodass es zu einer Wechselwirkung kommt falls ein Gammaquant die Scheibe trifft. Die Absorberdicke wird so klein gewählt, dass sich einzelne Scheiben nicht überlappen. Eine Skizze dieser Vorstellung ist in Abb. \ref{fig:sigma} dargestellt.
Für die Anzahl der Wechselwirkungen $d\text{N}$ über eine Dicke $d\text{x}$ und bei $N_0$ aufgetroffenen Gammaquanten folgt damit:
\begin{equation}
    d\text{N} = n N_0 \sigma d\text{x}
\end{equation}
mit der Elektronendichte des Absorbers $n$. Daraus ergeben sich für einen Absorber der Dicke $D$ mit
\begin{equation}
N(D) = N_0 \exp(-n \sigma D) \text{bzw.}  N_2(D) = N_0 (1 - \exp(-n \sigma D))
\end{equation}
die Anzahl der wechselgewirkten Quanten $N(D)$ bzw. die Anzahl der noch verbliebenen Quanten $N_2(D)$. Die Konstante $ \mu = n \sigma$ wird auch als Extinktionskoeffizient bezeichnet und ihr Kehrwert liefert die mittlere Reichweite der quanten im Absorber.
%wirkungsquerschnitt erläutern

\subsubsection{Photoeffekt}
Der Photoeffekt beschreibt die vollständige Absorption der Energie eines Gammaquants durch ein Hüllenelektron des Absorbers, welches daraufhin genug Energie besitz um den Wirkungsbereich seines Kernes zu verlassen. Dafür muss die Energie des Gammaquants $E_\gamma$ größer sein als die Bindungsenergie $E_\text{B}$ des Elektrons. Aufgrund von Impulserhaltung liegen die beteiligten Elektronen meistens in der K-Schale. Die überschüssige Energie des Gammaquants wird in kinetische Energie des nun freien Elektrons umgewandelt. Der Photoeffekt lässt sich daher durch 
\begin{equation}
    \gamma + \text{Atom} \to \text{Atom}^+ + e^-
\end{equation}
beschreiben. Das nun in der Schale entstandene Loch wird anschließend durch ein Elektron einer höheren Schale gefüllt, welche ihrerseits wieder aus noch höheren Schalen gefüllt werden.
%auch noch nich so geil(satz links)
Die dabei frei werdenden Energien werden in Form von Röntgenquanten abgegeben. Diese verlassen den Detektor im Normalfall nicht, wewegen die gesamte Energie des Gammaquants im Absorber bleibt.
%satz sieht geklaut aus  
Der Wirkungsquerschnitt des Photoeffektes lässt sich näherungsweise durch
\begin{equation}
    \sigma_\text{ph} \propto z^\alpha E^\delta
\end{equation}
beschreiben. Für die Energiebereiche natürliche Strahler lassen sich $\alpha$ und $\delta$ auf $4 < \alpha <5 $ und $\delta \approx -3.5$ eingrenzen. 

\subsubsection{Comptoneffekt}
Der Comptoneffekt beschreibt die unelastische Streuung eines Gammaquants an einem freien Elektron, wobei der Gammaquant einen Teil seiner Energie an das Elektron abgibt und seine Bewegungsrichtung ändert. Das Elektron wird hierbei als ruhende, punktförmige Masse betrachtet. Daher lässt er sich durch
\begin{equation}
    \gamma + e^- \to \gamma' + e^-
\end{equation}
beschreiben mit $E_{\gamma'} < E_\gamma$.
%Bild der comptonstreuung einfügen
Mit der Abkürzung
\begin{equation}
    \epsilon = \frac{E_\gamma}{m_e c^2}
\end{equation}
folgen für die kinetischen Energien des Gammaquants nach dem Stoß, beziehungsweise des gestoßenen Elektrons:
\begin{equation}
    E_{\gamma'} =  E_\gamma \frac{1}{1+ \epsilon (1-\cos(\theta_\gamma))} \text{bzw.} E_{e'} =  E_\gamma \frac{\epsilon (1-\cos(\theta_\gamma))}{1+ \epsilon (1-\cos(\theta_\gamma))} .
\end{equation}
Daraus ergibt sich unter einem Streuwinkel $\theta_\gamma$ von $\SI{180}{\degree}$ ein maximaler Energieübertrag auf das Elektron von
\begin{equation}
    E_{e,\text{max}} = \frac{2 \epsilon}{1 + 2 \epsilon} E_\gamma ,
\end{equation}
welcher weiterhin geringer ausfällt als die eigentliche Energie des Gammaquants. %klingt ziemlich ähnlich zur anleitung 
Aufgrund der Winkelabhängigkeit der auftretenden Energieüberträge ist der Comptoneffekt für die Gammaspektroskopie ein unerwünschter Effekt. Es zeigt sich, dass die Strahlung der Comptonstreuung keine isotrope Winkelverteilung besitzt. Für niederenergetische Gammaquanten zeigt sie Ähnlichkeiten zu einer Dipolstrahlung, bei hohen Energien oberhalb der Ruheenergie von Elektronen findet die Streuung vornehmlich in Vorwärtsrichtung statt. Für den niederenergetischen Fall mit $E_gamma < \SI{10}{\kilo\electronvolt}$ kann der Wirkungsquerschnitt näherungsweise durch den Thomsonschen Wirkungsquerschnitt dargestellt werden und erhält
\begin{equation}
    \sigma_\text{Co} = 2 \pi r_e^2
\end{equation}
mit dem klassischen Elektronenradius $r_e$. Da der maximale Energieübertrag des Comptoneffektes geringer ausfällt als der Peak des zugehörigen Photoeffektes, kommt es hinter dem maximalen Energieübertrag, der Comptonkante zu einer Lücke im Spektrum. Da ein Strahler in der Regel mehrere Zerfallskanäle mit unterschiedlichen Peaks besitzt, sind die Lücken im Experiment oft ohne geeignete Auswertung nicht zu erkennen.  
%vll bild von comptonkante +erklärung
\subsubsection{Paarbildung}
Im Falle hochenergetischer Gammaquanten mit $E_\gamma > 2 m_e c^2$ können sich diese in ein Elektron und ein Positron umwandeln. Da jedoch Impulserhaltung gilt muss ein geeigneter Stoßpartner vorhanden sein, welcher den den Impuls des Gammaquants aufnimmt. Dies ist meistens ein Atom, Elektronen sind jedoch auch möglich. Bei letzteren fällt die Rückstoßenergie, welche mit $m^{-1}$ läuft, allerdings deutlich größer weshalb für die Paarbildung mit einem Elektron $E_\gamma > 4 m_e c^2$ erfüllt sein muss. Die überschüssige Energie des Gammaquantes verteilt sich anschließend in Form von kinetischer Energie gleichmäßig auf Elektron und Positron. Auch für die Paarbildung lässt sich nur schwer ein allgemeiner Wirkungsquerschnitt abhängig von Kernladungszahl $Z$ und $E_\gamma$ angeben, da dieser aufgrund der auftretenden Coulombkräfte abhängig vom Ort in der Elektronenhülle ist, an welchem die Paarbildung stattgefunden hat. Es können jedoch Spezialfälle für verschwindende und vollständige Abschirmung der Kernkräfte angegeben werden. Für diese gilt:
\begin{equation}
    \sigma_\text{kernnah} = \alpha r_e^2 Z^2 \left( \frac{28}{9}\ln(2) \epsilon - \frac{218}{27}\right) \text{mit } \SI{10}{\mega\electronvolt} < E_\gamma \SI{25}{\mega\electronvolt}
\end{equation}

\begin{equation}
    \sigma_\text{kernfern} = \alpha r_e^2 Z^2 \left( \frac{28}{9}\ln\left(\frac{183}{\sqrt[3]{Z}}\right) \epsilon - \frac{218}{27}\right) \text{mit } \SI{500}{\mega\electronvolt} < E_\gamma
\end{equation}
Eine allgemeinere Form ist in Abb. \ref{fig:BILD EINFÜGEN} dargestellt.
%mherer paeks bei paarbildung erklären
Allerdings ist im 
Da die entstandenen Positronen jedoch wieder mit Elektronen im Detektor annihilieren zu  Gammaquant ausbildet, welcher den Detektor wieder verlassen kann


Zusammengefasst bilden die Wirkungsquerschnitte aller betrachteter Effekte im Falle vom Germanium einen Graphen proportional zu Abb. \ref{fig:BILD} aus. %Germaniumbild einfügen



\subsection{Die Wirkungsweise eines Reinst-Germaniumdetektors}
Den Kern des Reinst-Germaniumdetektors bildet eine Halbleiterdiode. Diese besteht aus zwei unterschiedlich dotierten Materialien, welche an einer Grenzfläche aneinandertreffen. Die Außenseite besteht aus einer mit Lithium stark n-dotierten Oberfläche, die Innenseite aus einer mit Gold stark p-dotierten Oberfläche. Aufgrund der endlichen Temperaturen diffundieren die beweglichen Elektronen und Löcher durch die Grenzfläche zwischen beiden Bereichen hindurch und rekombinieren im Anschluss. Es verbleiben die Akzeptoren der im p-dotierten und die Donatoren im n-dotierten Bereich, welche eine positive bzw. negative Ladung ausbilden. Sie erzeugen ein elektrisches Feld, welches die Bewegung der Ladungsträger wieder verhindert. Um die Grenzfläche bildet sich dadurch eine Zone aus, welche arm an Ladungsträgern ist. Der ganze Vorgang kann durch ein Potentialbild nach Abb. \ref{fig:POTENTIAL} veranschaulicht werden. Gelangt nun ein Gammaquant in den Detektor und setzt ein hochenergetisches Elektron in der Verarmungszone frei, so stößt dieses auf seinem Weg durch die Verarmungszone weitere Elektronen und hebt so weitere Elektronen aus dem Valenzband ins Leitungsband. Die entstehenden Löcher verhalten sich dabei wie postive Ladungen. Aufgrund des in der Verarmungszone vorherrschenden elektrischen Feldes rekombinieren die Ladungsträger jedoch nicht, sondern werden anhand ihrer Ladung getrennt. Dies resultiert in einem Ladungsimpuls, welcher proportional zur Energie des gestreuten Gammaquants ist, da die Anzahl der gehobenen Elektron-Loch Paare von der Energie des Quants abhängt und die Höhe des Ladungsimpuls von der Anzahl der Elektron-Loch Paare.
% diese sätze sind partiell echt mies     
Da die Wahrscheinlichkeit einer Wechselwirkung des Gammaquants mit den Elektrone exponentiell von der Dicke des Materials abhängt ist es nötig die Breite der Verarmungszone zu maximieren. Diese hängt von mehreren Faktoren ab:
\subsubsection{Möglichkeiten die Verarmungszone zu verbreitern}
Die Breite der Verarmungszone ist abhängig von den Dotierungen beider Bereiche. Für sie gilt:
\begin{equation}
    d_\text{n}^2 = \frac{2 \epsilon \epsilon_0}{\text{e}_0} (U_\text{D} + U) \frac{n_\text{A}}{n_\text{D}(n_\text{A}+n_\text{D})} bzw. d_\text{p}^2 = \frac{2\epsilon \epsilon _0}{\text{e}_0} (U_\text{D} + U) \frac{n_\text{D}}{n_\text{A}(n_\text{A}+n_\text{D})}
\end{equation}
 Um eine möglichst breite Verarmungszone zu erreichen wird daher eine sehr asymetrische Dotierung gewählt, weshalb die $d_\text{n}$ zwar nur sehr gering ausfällt, $d_\text{p}$ hingegen sehr groß, womit die ganze Zone allein durch $d_\text{p}$ angenähert werden kann. Da die Breite nun proportional zu $\frac{1}{sqrt{N_\text{A}}}$ verläuft wird ein hochreiner germaniumkristall verwendet. Eine zweite Möglichkeit die Breite zu vergrößern liegt darin, die anliegende Sperrspannung $U$ zu erhöhen. Hier wächst die Breite mit $\sqrt{U}$. Allerdings kann die Spannung nicht beliebig erhöht werden. Grund hierfür Elektronen die durch die endliche Temperatur thermisch gelöst werden und anschließend im anliegende Feld beschleunigt werden. Die entstehenden Ströme verschlechtern die Eigenschaften des Detektors. Um die thermischen Elektronen zu reduzieren werden der Detektor und die anliegenden Komponenten auf ca. $\SI{77}{\kelvin}$ abgekühlt. Unter dieser Temperatur kann eine Spannung von $5-\SI{6}{\kilo\volt}$ angelegt werden. Mit diesen Maßnahmen wird eine Verarmungsbreite von ca. $\SI{3}{\centi\meter}$ erreicht.
 
 
\subsection{Die Eigenschaften eines Halbleiterdetektors}%nich mehr ganz so mist
Eine wichtige Detektorkenngröße für die Spektroskopie von Gammaquanten ist sein Auflösungsvermögen. Dieses wird im wesentlichen aus der Halbwertsbreite aufgenommener Peaks einer monochromatischen Gammastrahlung bestimmt. Die Peaks werden hierzu durch Gausskurven approximiert. Ist der Abstand der Mittelwerte zweier benachbarter Peaks mindestens so groß wie die Halbwertsbreite, so können beide Peaks unterschieden werden. Ein weiterern Aspekt bildet die Höhe der Peaks. Unter der Annahme, dass ein Gammaquant der in der Verarmungszone wechselwirkt in dieser auch seine gesamte Energie abgibt, so folgt die Höhe des Peaks aus der Bindungsenergie und der Anzahl der erzeugten Elektron-Loch Paare. Es zeigt sich jedoch, dass diese nur unter Mitwirkung von Phononen entstehen und sich ein Teil der Energie somit auch auf die Phononen verteilt. Daher kann nicht mehr von unkorrelierten seltenen Ereignissen ausgegangen werden und für die Standardabweichung der resultierende Verteilung folgt.
\begin{equation}
\sigma = \sqrt{F \frac{E_\gamma}{E_\text{B}} }
\end{equation}
mit dem die Korrekturfaktor $F$. Da die Fluktuationen aufgrund der Elektron-Loch Paar Erzeugung jedoch teilweise wieder durch die Fluktuationen der Phononen kompensiert werden gilt $F < 1$. Im Fall von Germanium gilt $F \approx 0.1$. Bei Betrachtung großer Zählraten kann wieder eine Gaussverteilung angenommen werden und die Halbwertsbreite bestimmt sich zu:
\begin{equation}
\Delta E_\text{1/2} = \sqrt{8 \ln(2) F E_\gamma E_\text{B}}
\end{equation}
Die Halbwertsbreite wird jedoch durch weitere Rauscheffekte erhöht. Dazu gehören Rauschen aufgrund von Restverunreinigungen im Detektorkristall, aufgrund der angeschlossenen Verstärker und aufgrund von Feldimhomogenitäten, welche der aus der Bauweise des Detektors folgen. Die Effekte haben nur sehr schwache Korrelationen, sodass ihre zusätzlichen Halbwertsbreiten quadratisch zur Ursprünglichen addiert werden können. Auch in diesem Fall können die Rauscheffekte durch geringe Temperaturen minimiert werden. Im Fall der spannung müssen kompromisse eingegangen werden, da da Rauscheffekte unter beiden extremen auftreten.

Eine letzte wichtige Kenngröße ist die Effizienz eines Detektors. Sie gibt die Energieabhängigkeit für die Wahrscheinlichkeit eines erfolgreichen Nachweis eines Gammaquants an. Aufgrund des kontinuierlichen Spektrums des Comptoneffekts ist im Energiebereich unterhalb von $\SI{3}{\mega\electronvolt}$ nur der Photoeffekt von Relevanz. Für diesen zeigen sich theoretisch jedoch insbesondere bei hohen Energien nur geringe Nachweiswahrscheinlichkeiten. Im experiment fallen diese allerdings höher aus. 


\subsection{elektrische Komponenten}
Die im Detektor entstandenen Ladungsimpulse müssen nun noch verarbeitet werden. Hierzu gelangen die Impulse zunächst in einen Vorverstärker. In diesem wird eine Spannung erzeugt, welche proportional zum über die Zeit integrierten Strom ist. Damit die Spannung nach einem registrierten Impuls wieder auf Null zurückfällt und sich nicht über mehrere Impulse hin addiert muss sie nach dem Ende des Impulses wieder abfließen. Da ein paralell geschalteter Wiederstand jedoch zu verstärktem Rauschen führen würde wird eine optoelektronische Schaltung verwendet. Diese lässt den strom abfließen, sobald die Spannung eine obere Schwelle überschreitet. Um Rauscheffekte zu minimieren werden sowohl Detektor als auch Vorverstärker mit flüssigem Stickstoff auf $\SI{77}{\kelvin}$ gekühlt. Die Hochspannungsversorgung ist zudem an einen Temperaturwächter gekoppelt, da die Spannung bei höheren Temperaturen das Detektormaterial beschädigen würde. Da ein plötzlicher Spannungsabfall aufgrund von höheren Temperaturen jedoch den Vorverstärker beschädigen würde ist zudem ein RC-Glied mit großer Zeitkonstante integriert. Anschließend gelangt die Spannung über eine kapazitative Signalleitung mit niedriger Impedanz in den Hauptverstärker. Dieser verstärkt die Signale auf einen Spannungsbereich von $0 - \SI{10}{\volt}$. Es ist darauf zu achten, dass die Verstärkung zeitlich konstant bleibt und die Bandbreite des Verstärkers geeignet gewählt wird. Letzteres reduziert zusätzliches rauschen. Danach werden die Impulse in einem Analog zu Digital Konverter, kurz ADC, in digitale Signale umgewandelt.
