
\section{Diskussion}
\label{sec:Diskussion}
Im folgendem sollen die Ergebnisse und Vorgehensweisen aus der Auswertung diskutiert werden.

%eingehen warum werte herausgenommen wurden:\\
In dem Abschnitt \ref{subsec:Ausgleichs} wurden zum Fitten zuerst jeweils zwei Kanäle zu einem zusammen gefasst und danach die ersten $9$ und letzten $37$ Wertepaare herausgenommen.
Die Kanäle wurden zusammen gefasst, da die Messdauer zu kurz war, um eine ausreichende Statistik zu erreichen. Es ergaben sich daher Nulleinträge bei $t \to T_\text{S}$. Diese werden mit diesem Vorgehen eliminiert. Zu beachten ist, dass durch dieses Vorgehen Auflösung verloren geht. Die Auflösung wurde bei der Auswertung nicht beachtet, da sie für die gesuchten Größen eine untergeordnete Rolle spielt. Die letzten $37$ Wertepaare beziehungsweise $72$ Messkanäle wurden herausgenommen, da bei diesen keine Ereignisse gemessen werden können. Die Ursache hierfür ist aufgrund der maximalen Messzeit von $T_\text{S}=\SI{20}{\micro\second}$ nach oben begrenzte Kanalnummer der verwendeten Kanäle. Die ersten $9$ Messwerte beziehungsweise $18$ Messkanäle wurden aus verschiedenen Gründen herausgenommen. Die Ersten wurden herausgenommen, da der Aufbau bei geringen Messzeiten $T$ unter Umständen nicht wie gewünscht funktioniert. Eine Möglichkeit ist, dass aufgrund der Verzögerungsleitung zwischen der Koinzidenz und dem Univibrator, nicht zwischen Start- und Stopimpulsen mit geringerem zeitlichen Unterschied als $\SI{30}{\nano\second}$ unterschieden werden kann. Ein anderer Grund wäre, dass die Spannungen zwischen TAC und Vielkanalanalysator zu gering sind. Ein Indiz dafür wäre, dass  $17880$ Ereignisse am Stopimpulszähler, jedoch nur insgesamt $16217$ Ereignisse in allen Kanälen zusammen gezählt wurden. Die Wertepaaren mit einer leicht größeren Messzeit zeigen hingegen einen stark ausgeprägtes Peak (siehe Abbildung \ref{fig:zweite}). Für diesen gibt es mehrere mögliche Ursachen. Zu einem könnte es dadurch zustande kommen, dass leichte Spannungsschwankungen am TAC fälschlicherweise als Ereignis aufgefasst werden. Eine andere Ursache wäre, dass ein Impuls durch Reflexionen in den Kabeln doppelt gezählt wird.
Es folgt eine Diskussion der bestimmten Größen:

Der in Abschnitt \ref{subsec:Ausgleichs} bestimmte Wert für die Lebensdauer des Myons von $\tau=\SI{2.06(3)}{\micro\second}$ weicht um \SI{6(2)}{\percent} vom Literaturwert im Vakuum von $\tau_\text{lit}=\SI {2.1969811(22)}{\micro\second}$ \cite{ParticlePhysics}  ab. Eine Ursache hierfür ist, dass es sich um den Literaturwert im Vakuum handelt, während die im Experiment bestimmte Halbwertszeit in Materie bestimmt wurde. Da die Myonen in Materie Bindungen zu myonischen Atomen eingehen können und in diesen eine geringere Lebensdauer aufweisen, fällt die bestimmte Halbwertszeit geringer aus.  %Dies könnte durch einen systematischen Fehler der Zeitabstände der Doppelimpulse des Doppelimpulsgenerators auftreten, da diese eine verfälschte Zeitskala verursachen würden. Tatsächlich weicht der berechnete Wert von $\num{436.9(2)}$ Kanälen von der Anzahl der Kanäle, in denen Ereignisse gezählt worden sind, mit ungefähr $450$ um ca. \SI{3}{\percent} ab. 
%Es könnte jedoch auch durch eine an dem Univibrator ungenau eingestellte Suchzeit von nicht genau $\SI{20}{\micro\second}$ verursacht werden.

Der in Abschnitt \ref{subsec:Ausgleichs} bestimmte Wert für den Untergrund von $U_\text{fit}=\num{2.7(4)}$ weicht um \SI{12(13)}{\percent} vom im Abschnitt \ref{subsec:Berechnung} berechneten Wert von $U_\text{th}=\num{2.408(4)}$ nach unten ab. Unter Berücksichtigung der $ \sigma$-Umgebung des experimentellen Wertes wird der Theoriewert jedoch mit eingeschlossen. Der bestimmte Untergrund erscheint daher realistisch.

Bei der Bestimmung der Koinzidenzzeit wurde mit $\SI{-1.1(3)}{\nano\second}$ eine negative Zeit ermittelt. Es ist zu vermuten, dass dies durch eine an den Diskriminatoren eingestellte Impulsbreite von leicht mehr als $\SI{20}{\nano\second}$ zustande kommt. Da die Messzeit nur über ein externes Oszilloskop abgelesen werden konnte ist diese Vermutung realistisch. 




%sonst noch was??