\section{Auswertung}
\label{sec:Auswertung}


Die Graphen wurden sowohl mit Matplotlib \cite{matplotlib} als auch NumPy \cite{numpy} erstellt. Die
Fehlerrechnung wurde mithilfe von Uncertainties \cite{uncertainties} durchgeführt.


\subsection{Bestimmung der Zeitskalar}
\begin{figure}
	\centering
	\includegraphics[width=\linewidth-70pt,height=\textheight-70pt,keepaspectratio]{build/LinFit.pdf}
	\caption{Die Zeit $T_\text{gen}$ zwischen den Impulsen aus dem Doppelimpulsgenerator gegen die Messkanäle aufgetragen.}
	\label{fig:erste}
\end{figure}
\begin{table}
	\centering
	\caption{Die Kanäle mit den Peaks, wobei die Werte mit Fehler durch einen gewichteten Mittelwert zustande kommen.}
	\input{build/tab1.tex}
	\label{tab:b}
\end{table}
Für die Wertepaare aus Tabelle \ref{tab:tab1}  wurde eine lineare Ausgleichsrechnung durchgeführt. Die ermittelte Ausgleichsgerade und die Wertepaare sind in Abbildung \ref{fig:erste} dargestellt. Für die Parameter der Geraden $g(x)=a x + b$ ergibt sich
\begin{gather*}
a=\num{}\\
b=\num{}
\end{gather*}
und somit für den Umrechnungsfaktor von Kanälen zu \si{\micro\second}
\begin{displaymath}
u=\num{}.
\end{displaymath}

\subsection{Berechnung des Untergrundes mit Hilfe der gesamten Anzahl von Start-Ereignissen}
Mit der gesamten Anzahl von $\num{2212943}$ Start-Ereignissen in $\SI{92164}{\second}$ ergibt ein Erwartungswert $\lambda$ von $\num{1000}$ Start-Ereignissen in $\SI{20}{\micro\second}$. Mit der Annahme einer Poisson-Verteilung ergibt sich somit für die Wahrscheinlichkeit das genau ein Myon einem vorherigem in der Messzeit von $\SI{20}{\micro\second}$ folgt 
\begin{displaymath}
f=\lambda \exp(-\lambda)=\num{}.
\end{displaymath}
Damit folgt für die erwartete Anzahl an Untergrund-Ereignissen
\begin{displaymath}
\cdot f =\num{}.
\end{displaymath}
Mit der zuvor bestimmten Zeitskalar lässt sich die Anzahl der verwendeten Kanäle zu $\num{}$ berechnen.
Somit ergibt sich für die Anzahl der Untergrund-Ereignisse pro Kanal
\begin{displaymath}
=\frac{}{} =\num{}.
\end{displaymath}


\subsection{Bestimmung der Lebensdauer und des Untergrundes mit Hilfe einer nicht linearen Ausgleichsrechnung}
\begin{figure}
	\centering
	\includegraphics[width=\linewidth-70pt,height=\textheight-70pt,keepaspectratio]{build/Fit.pdf}
	\caption{Die Anzahl der gemessenen Ereignisse in den verschiedenen Messkanälen.}
	\label{fig:zweite}
\end{figure}
Die für den Fit in Abbildung \ref{fig:zweite} nötige nicht lineare Ausgleichsrechnung mit der Funktion curvefit aus der python-Bibliothek scipy \cite{scipy} durchgeführt. Die gefittete Funktion besitzt die Form $f(x)=\exp(-a x +b)+c$. Die ersten $ $ und letzen $ $ Wertepaare wurden für den Fit ignoriert. ???Die Messwertepaare sind in der Tabelle ??? zu finden.??? Für die Parameter ergibt sich somit
\begin{gather*}
a=\num{}\\
b=\num{}\\
c=\num{}.
\end{gather*}
Für den aus dem Fit ermittelten Untergrund ergibt sich somit
\begin{displaymath}
	=\num{}
\end{displaymath}
und für die Lebensdauer des Myons
\begin{displaymath}
	=\num{}.
\end{displaymath}