
\section{Theorie}
\label{sec:Theorie}
\subsection{Die Eigenschaften der  Myonen}
Nach heutigem Wissensstand besteht die bekannte Materie aus kleinsten Teilchen, den Elementarteilchen. Diese werden im Standardmodell der Elementarteilchen in Hadronen mit ganzzahligem Spin und Leptonen mit halbzahligem Spin eingeteilt. Zu den Leptonen gehören Elektronen, Myonen und Tauonen sowie ihre entsprechenden Antiteilchen. Zusätzlich besitzt jedes Lepton ein zugehöriges Neutrino. Diese sind aufgrund ihrer schwachen Wechselwirkung mit Materie jedoch nur schwer nachzuweisen. Die Leptonen unterscheiden sich hauptsächlich in ihrer Masse. Ein Myon wiegt $206 m_\text{e}$, ein Tauon $3491 m_\text{e}$. Im Experiment werden Myonen kosmischen Ursprung beobachtet. Sie sind ein Produkt von Pionenzerfällen, welche ca. in einer  Höhe \SI{10}{\kilo\meter} über der Erdoberfläche stattfinden. Für diese gilt:
\begin{equation}
    %wie setzt man den \text befehl in der equation umgebung ein, lukas fragen, antiteilchen fehlen noch 
    \mu^- \to e^- + \bar{\nu_e} + \nu_{\mu} \text{bzw.} \mu^+ \to e^+ + \nu_e + \bar{\nu_{\text{\mu}}}  \text{.}
\end{equation}  
Die Pionen stammen ihrerseits aus Kollisionen hochrelativistischer Protonen mit einzelnen Luftmolekülen der Atmosphäre. 

\subsection{Die statistische Verteilung der Myonenlebensdauern}
Der Zerfall von Myonen ist ein statistischer Prozess. Daher sind die Lebensdauern einzelner Teilchen nicht gleich, sondern über eine Verteilung bestimmt. Daher wird eine verallgemeinerte Lebensdauer über den Erwartungswert der Verteilung definiert. Aus der Überlegung, dass die Zerfallswahrscheinlichkeit nicht vom individuellen Alter einzelner Teilchen abhängt, folgt:
\begin{equation}
    \text{dN} = -N \text{dW} = - \lambda N \text{dt} \text{.}
\end{equation}
Damit lässt sich die Verteilung der Myonenlebensdauern
\begin{equation}
    \frac{\text{dN(t)}}{N_0} = \lambda \text{e}^{-\lambda t} \text{d}t \text{.}
\end{equation}
bestimmen.
Sie wird als Exponential-Verteilung bezeichnet. Der Erwartungswert liefert die verallgemeinerte Lebensdauer:
\begin{equation}
    \tau = \frac{1}{\lambda}\text{.}
\end{equation}
Daher wird $\lambda$ auch als Zerfallskonstante bezeichnet.