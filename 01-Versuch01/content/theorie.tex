
\section{Theorie}
\label{sec:Theorie}
\subsection{Eigenschaften der leptonischen Myonen}
Nach heutigem Wissensstand besteht sämtliche Materie aus sogenannten Elementarteilchen. Diese werden nach dem Standardmodell der Elementarteilchen in Hadronen mit ganzzahligem Spin und Leptonen mit halbzahligem Spin eingeteilt. Zu Letzteren gehören Elektronen, Myonen und Tauonen und ihre zugehörigen Antiteilchen. Sie können in Familien eingeteilt werden, welche die jeweils die gleichen Eigenschaften besitzen und sich hauptsächlich nur in ihrer Masse unterscheiden. Ein Myon wiegt ca. das 206 mal so viel wie ein Elektron, ein Tauon sogar 3491 mal so viel.  Zusätzlich besitzt jedes Lepton ein zugehöriges Neutrino mit entgegengesetzten Quantenzahlen. Aufgrund ihrer schwachen Wechselwirkung mit Materie sind Neutrinos jedoch nur schwer nachzuweisen. Die im Experiment beobachteten kosmischen Myonen sind Produkte von Pionzerfällen, welche ca. \SI{10}{\kilo\meter} über der Erdoberfläche stattfinden. Für diese gilt:
\begin{equation}
    %wie setzt man den \text befehl in der equation umgebung ein, lukas fragen, antiteilchen fehlen noch 
    \mu^- \to e^- + \bar{\nu_e} + \nu_{\mu} \text{bzw.}   .
\end{equation}  
Die benötigten Pionen wiederum entstehen aus Kollisionen hochrelativistischer Protonen mit einzelnen Luftmolekülen der Atmosphäre. 

\subsection{Die statistische Verteilung der Myonenlebensdauern}
Der Zerfall der kosmischen Myonen ist ein statistischer Prozess. Daher sind die Lebensdauern der individuellen nicht gleich, sondern über einen weiten Bereich gestreut. Es wird daher eine verallgemeinerte Lebensdauer definiert, welche sich aus dem Erwartungswert ergibt. Aus der Überlegung, dass die Zerfallswahrscheinlichkeit nicht vom individuellen Alter einzelner Teilchen abhängt, folgt:
\begin{equation}
    \text{dN} = -N \text{dW} = - \lambda \text{dt}.
\end{equation}
Damit lässt sich die Verteilung der Lebensdauern
\begin{equation}
    \frac{\text{dN(t)}}{N_0} = \lambda \text{e}^{-\lambda t} \text{d}t \text{.}
\end{equation}
Sie wird als Exponential-Verteilung bezeichnet. Die verallgemeinerte Lebensdauer berechnet sich nun über den Erwartungswert und liefert für die Exponential-Verteilung:
\begin{equation}
    \tau = \frac{1}{\lambda}\text{.}
\end{equation}
Daher wird $\lambda$ auch als Zerfallskonstante bezeichnet.