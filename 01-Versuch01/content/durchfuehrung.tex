
\section{Durchführung}
\label{sec:Durchführung}
%Die Durchführung besteht nun darin, jede einzelne Komponente des Versuchsaufbaus auf ihre Funktionalität zu prüfen und die unbekannten Parameter passend zu justieren. Hierzu werden die einzelnen Bauteilen einzeln durchgegangen. 
Zunächst wird der Aufbau gemäß Abb. $\ref{fig:Aufbau}$ bis zur Koinzidenz aufgebaut. Anschließend werden die hinter den Photomultipliern auftretenden Signale auf beiden Seiten mit einem Oszilloskop betrachtet.
% und es wird eine ungefähre Signallänge aufgenommen. 
Um Reflexionen zu vermeiden wird ein $\SI{50}{\ohm}$ Widerstand dazwischengeschaltet. Dies ist bei allen weiteren Benutzungen des Oszilloskop zu beachten. Der Vorgang wird nochmals hinter den Diskriminatoren wiederholt. Nun sollten die Impulse eine einheitliche Länge und Höhe besitzen. An den Diskriminatoren wird eine Impulsdauer von $\SI{20}{\nano\second}$ eingestellt. Zum justieren wird wieder das Oszilloskop verwendet.
% Auch diese Informationen werden notiert.
 Im Anschluss wird das Ossziloskop gegen zwei Zählwerke ausgetauscht. Der Treshold der Diskriminatoren wird so eingestellt, dass die mittlere Rate der eintreffenden Impulse auf beiden Seiten bei ca. $\SI{40}{\frac{1}{\second}}$ liegt. Nun werden die Verzögerungsleitungen angepasst. Hierzu wird ein Zählwerk hinter die Koinzidenzapperatur geschaltet und eine Messkurve unter der Impulsraten unter Variation der Verzögerung erstellt. Die Kurve wird jedoch kein eindeutiges Maximum besitzen. Daher werden die Wendepunkte der Kurve ermittelt und aus ihrem Mittelwert das Maximum bestimmt.Die Verzögerungsleitungen werden auf das Maximum eingestellt. Lässt sich keine Verringerung der Impulsrate gegenüber den Messungen vor der Koinzidenzapperatur feststellen, müssen die Tresholds weiter verringert und der Messvorgang widerholt werden. Nun wird die eigentliche Messchaltung nach Abb. $\ref{}fig:Aufbau$ aufgebaut. Danach wird die Zeit $T_\text{S}$ am Univibrator mithilfe eines Oszilloskops auf $\SI{20}{\micro\meter}$ eingestellt. Zur weiteren Justierung werden beide Signalgänge des Szintillators abgeklemmt und gegen einen Doppelimpulsgenerator ausgetauscht. Es sollten mithilfe des Oszilloskops am Start und Stopp Eingang des TAC's Impulse der eingestellten Länge zu erkennen sein. Am TAC wird eine Messzeit von $\SI{20}{\micro\second}$ eingestellt. Das Ossziloskop wird hinter den TAC geschaltet. Es sollten nun Impulse zu erkennen sein, deren Höhen proportional zum eingestellten Impulsabstand verlaufen. Anschließend werden die Kanäle des Vielkanalanalysators am Computer betrachtet. Im verwendeten Programm sollte  ein Punkt zu sehen sein, der sich vertikal nach oben bewegt. Zuletzt wird noch eine Zeitkalibrierung durchgeführt. Hierzu wird der Doppelimpulsabstand variiert und die am Computer entstehende Kurve gespeichert. Aus den Daten lässt sich später eine Zeitskala für die Kanäle anfertigen.Nun kann die eigentliche Messreihe gestartet werden. Hierfür wird die Koinzidenz wieder auf die Impulse des Szintillators angepasst. Dann wird die Messung gleichzeitig am Computer und an beiden Impulsmessern gestartet. Nach ca. $24-30$ Stunden kann die Messung wieder beendet werden. Auch hier ist darauf zu achten, dass gleichzeitig gestoppt wird.  