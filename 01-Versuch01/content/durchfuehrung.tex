
\section{Durchführung}
\label{sec:Durchführung}
Die Durchführung besteht nun darin, jede einzelne Komponente des Versuchsaufbaus auf ihre Funktionalität zu prüfen und die unbekannten Parameter passend zu justieren. Hierzu werden die einzelnen Bauteilen einzeln durchgegangen. Zunächst wird der Aufbau gemäß des Schaltplanes bis zur Koinzidenz aufgebaut. Anschließend werden die hinter den Photomultipliern auftretenden Signale auf beiden Seiten des Szintillators mit einem Oszilloskop betrachtet und eine ungefähre Signallänge aufgenommen. Um Reflexionen zu vermeiden wird ein $\SI{50}{\ohm}$ Widerstand dazwischengeschaltet. Dies ist bei allen weiteren Benutzungen des Oszilloskop zu beachten. Dieser Vorgang wird nochmals hinter den Diskriminatoren wiederholt. Nun sollten die Signale eine einheitliche Länge und Höhe besitzen. An den Diskriminatoren wird nun eine Impulsdauer von $\SI{20}{\nano\second}$ eingestellt. Zur Justierung wird das Oszilloskop verwendet. Auch diese Informationen werden notiert. Im Anschluss wird das Ossziloskop gegen zwei Zählwerke eingewechselt. Der TRESHOLD der Diskriminatoren wird nun so justiert, sodass die mittlere Rate der eintreffenden Impulse auf beiden Seiten bei ca. $\SI{40}{\frac{1}{\second}}$ liegt. Nun wird die Koinzidenz eingestellt. Hierzu wird eine Messkurve aufgenommen für welche die hinter der Koinzidenz auftretenden Impulse unter Variation der Verzögerungsleitung mit einem Zählwerk gemessen werden. Da die Kurve eine Form nach Abb. 
%Bild einer typischen messkruve
hat, lässt sich kein eindeutiges Maximum feststellen. Daher werden die Wendepunkte der Kurve ermittelt und aus ihrem Mittelwert das Maximum bestimmt.Die Verzögerungsleitungen werden auf das Maximum eingestellt. Lässt sich keine Verringerung der Impulsanzahl gegenüber den Messungen vor der Koinzidenzapperatur feststellen, müssen die THRESHOLDS verringert werden und der Messvorgang widerholt werden. Nun wird der Aufbau unterhalb der Koinzidenz aufgebaut. Danach wird die Zeit $T_\text{S}$ am Univibrator mithilfe eines Oszilloskops auf $\SI{20}{\micro\meter}$ eingestellt. Zur weiteren Justierung werden beide Signalgänge des Szintillators abgeklemmt und gegen die Impulse eines Doppelimpulsgenerators ausgetauscht. Es sollten am Start und Stopp Eingang des TAC Impulse mit der eingestellten Länge mithilfe des Osszilokops zu erkennen sein. Am TAC wird eine Messzeit von $\SI{20}{\micro\second}$ eingestellt. Das Ossziloskop wird hinter den TACgeschaltet. Es sollten nun Impulse zu erkennen sein, deren Höhen proportional zum eingestellten Impulsabstand liegen. Anschließend werden die Kanäle des Vielkanalanalysators am Computer betrachtet. Bei diesen sollte sich ein Punkt vertikal nach oben bewegen. Zuletzt wird noch eine Zeitkalibrierung durchgeführt. Hierzu wird der Doppelimpulsabstand variiert und die am Computer entstehende Kurve gespeichert. Aus den Daten lässt sich später eine Zeitskala für die Kanäle anfertigen.Nun kann die eigentliche Messreihe gestartet werden. Hierfür wird die Koinzidenz wieder auf die Impulse des Szintillators eingestellt. Dann wird die Messung am Computer gleichzeitig mit beiden Impulsmessern gestartet. Nach ca. 24-30 Stunden kann die Messung wieder beendet werden. Auch hier ist darauf zu achten, dass der Stoppvorgang gleichzeitig geschieht.  