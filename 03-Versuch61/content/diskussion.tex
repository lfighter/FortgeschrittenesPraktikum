
\section{Diskussion}
\label{sec:Diskussion}
Der in Abschnitt \ref{ssec:polar} bestimmte Wert für den optimalen Winkel des Polarisationsfilters weicht mit $\phi_\text{max}=\SI{48.1(4)}{\degree}$ um $\SI{3.1(4)}{\degree}$ vom theoretischen Winkel von $\SI{45}{\degree}$ nach oben ab. Dies könnte zu einem daran liegen, dass der Winkel der am Polarisationsfilter eingestellt ist nicht dem Polarisationswinkel des Lichtes hinter dem Filter entspricht. Auch könnte der dahinter liegende PBSC leicht gedreht sein oder einer der zwei polarisierten Strahlen durch eventuelle Verunreinigungen zum Beispiel an den Spiegeln in der Intensität gesenkt werden, sodass die Intensität der Strahlen durch einen leicht anderen Winkel angepasst werden muss. 



Der im Abschnitt \ref{ssec:glas} bestimmte Wert für den Brechungsindex vom Glas $n_\text{Glas}=\num{1.520(7)}$ stimmt gut mit dem typischen Brechungsindex von Kronglas $n_\text{Kron}=\num{1.523}$ \cite{nKron} überein, was vermuten lässt, dass es sich bei dem Material der Glasplatten um Kronglas handelt. Eine mögliche Ursache für eine Abweichung vom tatsächlichen Wert ist, dass der Winkel zu schnell oder ungleichmäßig verändert wurde, sodass Maxima nicht oder mehrfach gezählt wurden.

Die Messwerte in Abbildung \ref{fig:Luft} in Abschnitt \ref{ssec:luft} liegen gut auf der bestimmten Ausgleichsgerade. Sie besitzen eine Abweichung zu dieser von $\sigma_y=\num{5.2e-6}$. Der im selben Kapitel bestimmte Wert für den Brechungsindex von Luft $n_\text{Luft}=\num{1.000266(2)}$ bei \SI{1013}{\milli\bar} und \SI{26.5}{\degreeCelsius} weicht um \SI{1(2)e-4}{\percent} vom Literaturwert $n_\text{lit}=\num{1.000265205(32)}$ \cite{nist} nach oben ab. Der ermittelte Wert und der Literaturwert stimmen also sehr gut überein, woran noch einmal die hohe Präzision aufgrund der Stabilität des Sagnac-Interferometers deutlich wird.


