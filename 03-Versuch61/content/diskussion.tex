
\section{Diskussion}
\label{sec:Diskussion}
Die Auswertung hat einige Ergebnisse erbracht, welche nun noch zu diskutieren sind. Die anfängliche Überprüfung der Stabilitätsprüfung konnte zumindest für den Fall unterschiedlicher Resonatorspiegel zeigen, dass ein Strahl nicht stabil oberhalb der vorhergesagten Maximallänge bestehen kann und bestätigt damit die Theorie. Für den zweiten Resonatoraufbau bestehend aus zwei sphärischen Spiegeln kann jedoch nur gesagt werden, dass sich die Kurve des Stabilitätskriteriums von dem bestehend aus einem planarem und einem sphärischen Spiegel unterscheidet. Über die obere Grenze kann jedoch keine Aussage getroffen werden. %Diese Behauptung erfüllt sich jedoch nicht für den Fall gleicher Spiegel. Da der Aufbau nicht lang genug gewesen ist, kann eine solche Aussage nicht getroffen werden. Es kann jedoch gesagt werden, dass sich das Verhalten beider Resonatoren unterscheiden muss. Eine genauere Einschätzung ist aber nicht möglich.
 Nun folgt eine Beurteilung der vermessenen $\text{TEM}_\text{00}$ Mode in Abb. \ref{fig:T00}. An dieser ist zu erkennen, dass die Mode um ca. $c = \SI{5.28(6)}{\milli\meter}$ zur Nullmarkierung der Mikrometerschiene verschoben ist. Die Messwerte liegen größtenteils in Nähe zur bestimmten Fitkurve und zeigen die vorhergesagte Form einer Glockenkurve. Die Abweichungen insbesondere im Bereich um $\SI{8}{\milli\meter}$ können zumindest zum Teil durch Schwankungen der Intensität erklärt werden. Ein Grund hierfür bildet der zur Stabilisierung der Mode eingesetzte Draht. Da dieser fast so breit wie der produzierte Laserstrahl war, kam nur wenig Licht hindurch und durch kleine Störungen konnte der Draht die Intensität kurzzeitig noch weiter schwächen. Dieser Effekt wird noch deutlicher bei der anschließenden Vermessung der $\text{TEM}_{01}$ Mode. Diese konnte während der Durchführung auch mithilfe des Drahtes nur schwer stabil erzeugt werden und die Intensitäten zeigten deutliche Schwankungen während des Messvorgangs. Jedoch zeigen auch die Ergebnissen der $\text{TEM}_{01}$ Mode den vorhergesagten Verlauf. Es ist jedoch zu erkennen, dass die Messwerte in Abb. \ref{fig:T01} im Bereich vor dem lokalen Minimum des Fits unterhalb des Fits liegen, im Bereich hinter dem Minimum jedoch über dem Fit liegen. 
%erklärung für diese verschiebung überlegen


Die Vermessung der Polarisation des Laserstrahls zeigt nach Abb. \ref{fig:polarisation}, dass die Messwerte gut durch $\cos(\phi)^2$ dargestellt werden können. Dies zeigt sich auch in den Fehlern der bestimmten Parameter, welche nur wenige Prozent betragen. Die beiden größeren Abweichungen können als Ausreißer behandelt werden. Aus dem dem bestimmten Parameter $c = \num{1.20(4)}$ folgt zudem, dass die maximale Intensität bei einem Winkel von ca. $\SI{69}{\degree}$ liegt und nicht wie vorher angenommen bei einem Winkel von $\SI{45}{\degree}$. Ursache hierfür ist vermutlich eine zur Justierung durchgeführte Drehung des Laserrohrs und der daran befestigten Brewsterfenster. Desweiteren ist auch eine falsche Beschriftung der Skala des Polarisationsfilters möglich.
