
\section{Diskussion}
\label{sec:Diskussion}
Die Auswertung hat einige Ergebnisse erbracht, welche nun noch zu diskutieren sind. Die anfängliche Überprüfung der Stabilitätsprüfung konnte zumindest für den Fall (flat,$\SI{140}{\centi\meter}$) zeigen, dass ein Strahl nicht stabil oberhalb der vorhergesagten Maximallänge bestehen kann und bestätigt damit die Theorie. Für den zweiten Resonatoraufbau ($\SI{140}{\centi\meter}$,$\SI{140}{\centi\meter}$) kann jedoch nur gesagt werden, dass sich die Kurve des Stabilitätskriteriums von dem bestehend aus einem planarem und einem sphärischen Spiegel unterscheidet. Über die obere Grenze kann jedoch keine Aussage getroffen werden. %Diese Behauptung erfüllt sich jedoch nicht für den Fall gleicher Spiegel. Da der Aufbau nicht lang genug gewesen ist, kann eine solche Aussage nicht getroffen werden. Es kann jedoch gesagt werden, dass sich das Verhalten beider Resonatoren unterscheiden muss. Eine genauere Einschätzung ist aber nicht möglich.
 Nun folgt eine Diskussion der vermessenen Moden beginnend mit der $\text{TEM}_\text{00}$ Mode in Abb. \ref{fig:T00}. An dieser ist zu erkennen, dass die Mode um ca. $c = \SI{5.28(6)}{\milli\meter}$ zur Nullmarkierung der Mikrometerschiene verschoben ist. Die Messwerte liegen größtenteils in Nähe zur bestimmten Fitkurve und zeigen die vorhergesagte Form einer Glockenkurve. 
 
 %T01
 Anschließend wird auch der bestimmte Graph der $\text{TEM}_\text{01}$ Mode in Abb. \ref{fig:T01} nochmals betrachtet. Es fällt auf, dass die Abweichungen der bestimmten Messwerte stärkere Abweichungen bezüglich des Fittes aufzeigen als die der $\text{TEM}_\text{00}$ Mode in Abb. \ref{fig:T00}. Eine Ursache hierfür ist höhere Instabilität der $\text{TEM}_\text{01}$ Mode gegenüber kleinen Störungen, welche sich auch nicht unbedingt durch den Stabilisierungsdraht verbessert hat. Ein möglicher Grund hierfür ist die Dicke des Drahtes, welcher fast so breit wie der Laserstrahl selbst ist. Daher fallen die bestimmten Intensitäten der $\text{TEM}_\text{00}$ Mode in Tabelle \ref{tab:T00} auch größer aus, als die der $\text{TEM}_\text{01}$ Mode in Tabelle \ref{tab:T01}.
 Allerdings werden auch die Ergebnisse der $\text{TEM}_{01}$ Mode gut durch den den vorhergesagten Verlauf beschrieben.  
%erklärung für diese verschiebung überlegen


Die Vermessung der Polarisation des Laserstrahls zeigt nach Abb. \ref{fig:polarisation}, dass die Messwerte gut durch $\cos(\phi)^2$ dargestellt werden können. Dies zeigt sich auch in den Fehlern der bestimmten Parameter, welche nur wenige Prozent betragen. Die beiden größeren Abweichungen können als Ausreißer behandelt werden. Aus dem dem bestimmten Parameter $c = \num{1.20(4)}$ folgt zudem, dass die maximale Intensität bei einem Winkel von ca. $\SI{69}{\degree}$ liegt und nicht wie im Vergleich zum Interferometer aus V64 angenommen bei einem Winkel von $\SI{45}{\degree}$. Ursache hierfür ist vermutlich eine zur Justierung durchgeführte Drehung des Laserrohrs und der daran befestigten Brewsterfenster. Desweiteren ist auch eine falsche Beschriftung der Skala des Polarisationsfilters möglich.


%wellenlänge
Zuletzt wird die bestimmte Wellenlänge diskutiert.
Bei Betrachtung des Graphens in Abb. \ref{fig:welle} ist zu erkennen, dass zwischen den einzelnen Wellenlängen Differenzen bestehen, je nachdem welches Maximum als Ausgangswert für die Rechnung genommen worden ist. Ein möglicher Grund hierfür sind Ungenauigkeiten während des Messprozesses, welche sich aufgrund der geringen Skalen des Messaufbaus noch auf das Ergebnis auswirken. Ein Vergleich mit der Literatur \cite{VHeNeGoettingen} zeigt, dass ein Helium Neon Laser keine eindeutige Wellenlänge besitzt sondern das diese in einem Wellenlängenbereich von $610-\SI{640}{\nano\meter}$ liegen. Damit liegt die bestimmte mittlere Wellenlänge von $c = \SI{620}{\nano\meter}$ im Bereich der Literaturwerte. Jedoch liefert die Auswertungsmethode keinen Fehler, weswegen der bestimmte Wert zwar durchaus realistisch ist, jedoch nicht sonderlich aussagekräftig.  