
\section{Durchführung}
\label{sec:Durchführung}
Die Durchführung besteht zum Großteil daraus, die einzelnen Spiegel des Interferometers geeignet auszurichten. %sodass sich beide Strahlen hinreichend genau überlappen.
Zuerst werden der HeNe Laser, die zugehörigen Lenkspiegel $M_\text{1}$ und $M_\text{2}$, der Polarisationsfilter, der PBSC und der erste Spiegel $M_\text{A}$ gemäß Abb. \ref{fig:Aufbau} auf dem Konstruktionstisch montiert. Der vom PBSC umgelenkte Strahl kann durch eine Abschirmung blockiert werden. Anschließend werden zwei Lochblenden an die Positionen 2 und 3 in den zweiten Strahlengang gebracht und die Spiegel $M_\text{1}$ und $M_\text{2}$ werden so justiert, sodass der Strahl beide Blenden passiert und auf $M_\text{A}$ trifft. Um spätere Justierungen zu vermeiden ist darauf zu achten, dass der Strahl bereits nach $M_\text{1}$ und $M_\text{2}$ möglichst horizontal verläuft.  Nun wird der Spiegel $M_\text{C}$ montiert. Hierzu werden die Lochblenden an die Positionen 8 und 9 gebracht und die Abschirmung wird vor den geraden Strahl gesetzt. 
%kein plan wie die das bei schritt 2 mit der halterung meinen, dann verdreh ich doch meinen graden strahl wieder
Zur Justierung werden die Schrauben der PBSC-Halterung gelockert und diese wird so gedreht, dass der Strahl beide Blenden passiert. Nun wird auch der letzte Spiegel $M_\text{B}$ montiert. Dazu werden zunächst die Lochblenden an die Positionen 5 und 6 gebracht. Anschließend wird zunächst $M_\text{A}$ so justiert, sodass der Strahl beide Lochblenden passiert. Um den Spiegel vertikal zu verstellen ist es notwendig kleine Metallscheiben unter den Spiegel zu klemmen. Das gleiche wird für den Spiegel $M_\text{C}$ wiederholt. Dieser lässt sich jedoch in der Höhe verstellen. Nach Entfernung der Blenden, sollte $M_\text{B}$ von zwei separaten Strahlen getroffen werden, die sich in einem Punkt auf der Spiegeloberfläche treffen.
%bis hier sind alle spiegel erstmals justiert
Die Spiegel $M_\text{A}$ und $M_\text{C}$ können nun festgestellt werden. Bevor auch $M_\text{B}$ festgestellt wird, wird vorher überprüft, dass die Strahlen ihn auch von beiden Seiten treffen. Um die Positionen der Strahlengänge zu erfassen, kann ein Stück Papier verwendet werden. $M_\text{B}$ wird nun so gedreht, sodass sich beide Strahlen überlappen. Nun kann auch $M_\text{B}$ festgestellt werden.
% Stelle danach die Abschirmung hinter die vierte Seite des Strahlteilers. 
Im Regelfall sollte der zusammengeführte Strahl sich noch nicht überlappen. Zur Feinjustierung können die Justierschrauben an $M_\text{A}$ und $M_\text{C}$ verwendet werden. Danach wird nochmals nachjustiert um auch noch mögliche Interferenzmuster im Strahl zu minimieren.
% Der erzeugte Strahl sollte nun einen Überlapp besitzen, jedoch werden keine Beugungsmuster zu erkennen sein.
 %Um die Ursache zu erkennen wird ein Polaroid zwischen Strahlteiler und Beobachtungsschirm gesetzt und der Strahl wird wieder leicht aufgetrennt. Ist die Polarisierung klar geworden wird der Polaroid in einem $\SI{45}{\degree}$ Winkel befestigt, sodass beide Strahlen zu Teil hindurchgelangen. Danach werden beide Strahlen wieder zum Überlapp gebracht. Zuletzt wird versucht die Ortsfrequenz der Beugungsmuster zu minimieren.
  Nun werden beide Strahlen wieder räumlich getrennt.
  Hierzu werden die Lochblenden wieder auf die Positionen 2 und 3 gebracht und $M_\text{2}$ wird über die Laufschiene soweit verschoben, bis der Strahl nicht mehr durch die mittleren Blendenöffnungen fällt, sondern durch die Seitlichen. Gegebenenfalls müssen dazu auch $M_\text{1}$ und $M_\text{2}$ in der Höhe verstellt werden. Es ist hilfreich den umgelenkten Strahl während der Justierung zu blockieren.
%alles nach stelle 8
Nun werden die Lochblenden wieder entfernt und es sollten im Interferometer zwei räumlich getrennte Strahlen zu erkennen sein.
%schritt 16
Nun wird die Halterung zur Bestimmung des Brechungsindex von dünnen Schreiben in die Strahlengänge der zwei parallelen Strahlen gebracht. Die integrierten Glasscheiben sind so angebracht, sodass sich beide unter einer Drehung in gegenläufige Richtungen drehen. Dies erhöht die erkennbare Phasendifferenz. Unter Rotation bilden sich nun gleichmäßig ändernde Beugungsmuster aus, welche bei einem Rotationsstopp jedoch stabil gegenüber äußeren Störungen bleiben.
%yeah
Um die Beugungsmuster zu optimieren kann wieder an den Daumenschrauben an $M_\text{A}$ und $M_\text{C}$ justiert werden.

%schriit 11
%da nun gezeigt worden ist, dass sich Beugungsmuster mithilfe eines Polaroids erzeugen lassen, wird dieses gegen eine effizientere Kombination ausgetauscht.
 
 Nun wird ein zweiter PBSC im $\SI{45}{\degree}$ Winkel hinter dem Ausgang des Sagnac Interferometers gemäß Abb. \ref{fig:Aufbau} montiert. Da der Strahl wieder in diesem geteilt wird, wird ein weiterer Spiegel hinter den umgelenkten Strahl gesetzt, wodurch beide Strahlen wieder in diesselbe Richtung laufen. Zur besseren Kontrastanalyse werden nun noch zusätzlich Photodioden hinter beiden Strahlen gesetzt. An beiden Dioden wird der gleiche Gain eingestellt.
 %kontrastanalyse
 Anschließend wird zunächst eine Kontrastmessung durchgeführt. Hierzu wird zunächst eine Photodiode an das Oszilloskop angeschlossen. Danach wird der Winkel des Polarisationsfilters variiert. Unter jedem eingestellten Winkel werden nun die dünnen Glasscheiben rotiert und es werden jeweils die auftretenden minimalen und maximalen Spannungen notiert. Zusätzlich wird vorher noch eine Spannungsmessung ohne Laser durchgeführt um den Untergrund aufgrund der restlichen Raumbeleuchtung zu ermitteln. Danach wird aus den Daten und mithilfe von Formel \ref{eq:kont} eine Kurve geplottet und der Polarisationsfilter wird auf den Winkel eingestellt unter welchem der maximale Kontrast auftritt. Zum Fortfahren sollte der maximale Kontrast bei über $0.9$ liegen. Ansonsten müssen die Spiegel nochmals angepasst werden. 
 % Der hier grade Strahl entspricht dem Überlapp und den im Interferometer auftretenden Interferenzen. Daher sollten bei Hinzunahme einer Leinwand Beugungsmuster zu erkennen sein.
  %Der umgelenkte Strahl wird über einen weiteren Spiegel ebenfalls auf die Leinwand gebracht. Anschließend wird das Interferometer so eingestellt, sodass an beiden Auftreffpunkten keine Ortsfrequenz mehr zu erkennen ist. Die Lichtintensitäten sollten nun zwischen beiden Punkten komplementär sein. Danach wird die Leinwand gegen
Nun fehlt nur noch ein Aufbau um die Interferenzen besser analysieren zu können.
%brechungsindex von glas
 Dazu werden beide Intensitäten über ein Zweikanaloszilloskop ausgelesen. Auch hier sollte das komplementäre Verhalten zwischen beiden Strahlen zu erkennen sein. Um eine bessere Darstellung zu erhalten wird die Differenz zwischen beiden Signalen betrachtet. Dazu wird ein Signal-Fluss-Schema dazwischengeschaltet. Es sollte zu erkennen sein, dass das Signal nur sehr kleinen Untergrund besitzt. Da im Interferometer jedoch weiterhin zwei parallele Strahlen verwendet werden, kommt es zu Anfälligkeiten gegenüber Luftturbolenzen. Es wird daher zusätzlich ein Windschutz über das Interferometer gestellt. Es ist nun stabil gegenüber Luftturbolenzen und Vibrationen. Da nun alle Komponenten justiert sind, kann die eigentliche Messung begonnen werden. Es wird mit der Bestimmung des Brechungsindex der Gläser begonnen. Dazu werden die Gläser langsam aber konstant mithilfe der Führung um $\SI{10}{\degree}$ rotiert und es werden die auftretenden Minima und Maxima gezählt. Dies wird mehrfach wiederholt. Anschließend wird der Brechungsindex von Luft bestimmt. Hierzu wird die Glashalterung gegen die Gaszelle  nach Abb. \ref{fig:Gas} ausgetauscht. Es ist darauf zu achten, dass die Zelle nur in einem der Strahlen liegt. Der Bleed wird geschlossen. Nun wird der Druck in der Gaszelle mit einer Pumpe auf wenige $\si{\milli\bar}$ gesenkt, das Ventil zur Gaszelle wird geschlossen und der Bleed wird anschließend wieder geöffnet. Zusätzlich werden die aktuelle Temperatur und der aktuelle Druck notiert. Danach wird auch das Ventil zu Gaszelle langsam geöffnet, sodass sich sinusförmige Spannungen auf dem Oszilloskop aufgrund der Interferenzeffekte zeigen. Alle $\SI{100}{\milli\bar}$ wird die Anzahl der bis dahin aufgetretenen Maxima notiert. Dies wird drei mal wiederholt. Die Messung wird noch mehrfach wiederholt, diesmal werden jedoch nur die Maximazahlen nach Wiedererreichen des Normaldrucks notiert. 