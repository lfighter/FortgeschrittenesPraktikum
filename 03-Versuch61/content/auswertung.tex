\section{Auswertung}
\label{sec:Auswertung}
Die Graphen wurden sowohl mit Matplotlib \cite{matplotlib} als auch NumPy \cite{numpy} erstellt. Die
Fehlerrechnung wurde mithilfe von Uncertainties \cite{uncertainties} durchgeführt.

\subsection{Überprüfung der Stabilitätsbedingung des Lasers}
Zunächst wird das in der Theorie angesprochene Stabilitätskriterium untersucht. Das Kriterium für einen stabilen Strahl im Experiment waren die sichtbare Streuung des Strahles an kleinen Staubkörnern in der Luft und die Stabilität des Strahles gegenüber kleinen Erschütterungen.
Hierbei ergibt sich auf Basis von \ref{eq:STABILITATFORMEL} und dem zugehörigen Graphen in Abb. \ref{fig:stabilitat} für einen Resonator bestehend aus einem flachen Spiegel und einem konkaven Spiegel mit einem Krümmungsradius von $\SI{140}{\centi\meter}$ eine theoretische maximale stabile Resonatorlänge $L$ von $\SI{140}{\centi\meter}$. Diese maximale Resonatorlänge konnte auch im Messaufbau nachgewiesen werden. Für einen Resonator bestehend aus zwei konkaven Spiegeln mit einem Krümmungsradius von $\SI{140}{\centi\meter}$ konnte die theoretische maximale Länge von $\SI{280}{\centi\meter}$ nach Abb. \ref{fig:stabilitat} aufgrund der Größe des Aufbaus nicht nachgeprüft werden, es konnte jedoch ein stabiler Strahl bei einer Länge von $\SI{194}{\centi\meter}$ erzeugt werden. Daher ist zumindest das unterschiedliche Stabilitätsverhalten beider Resonatortypen experimentell bestätigt worden.



\begin{figure}
	\centering
	\includegraphics[width=\linewidth-100pt,height=\textheight-100pt,keepaspectratio]{build/stabilitat.pdf}
	\caption{Die Graphen des Stabilitätskriterium für die zwei untersuchten Resonatorkombinationen in Abhängigkeit der Resonatorlänge $L$ .}
	\label{fig:stabilitat}
\end{figure}




\subsection{Vermessung der T00 Mode}


\begin{figure}
	\centering
	\includegraphics[width=\linewidth-100pt,height=\textheight-100pt,keepaspectratio]{build/T00.pdf}
	\caption{Der gemessenen Stromstärken im Verhältnis zur kleinsten gemessenen Stromstärke der $\text{TEM}_{00}$ Mode entlang der Horizontalen $x$ der Mode.Der Nullpunkt ist an die in Strahlrichtung linke Seite des Modenmusters gelegt.}
	\label{fig:T00}
\end{figure}



\subsection{Vermessung der T01 Mode}



\begin{figure}
	\centering
	\includegraphics[width=\linewidth-100pt,height=\textheight-100pt,keepaspectratio]{build/T01.pdf}
	\caption{Der gemessenen Stromstärken im Verhältnis zur kleinsten gemessenen Stromstärke der $\text{TEM}_{01}$ Mode entlang der Horizontalen $x$ der Mode.Der Nullpunkt ist an die in Strahlrichtung linke Seite des Modenmusters gelegt.}
	\label{fig:T01}
\end{figure}



\subsection{Vermessung der Polarisation des Laserstrahles}


\begin{figure}
	\centering
	\includegraphics[width=\linewidth-100pt,height=\textheight-100pt,keepaspectratio]{build/Polarisation.pdf}
	\caption{Der gemessenen Stromstärken im Verhältnis zur kleinsten gemessenen Stromstärke in Abhängigkeit des Winkels des Polarisationsfilters.}
	\label{fig:polarisation}
\end{figure}




\subsection{Bestimmung der ausgesendeten Wellenlänge des HeNe Lasers}




\begin{table}
	\centering
	\caption{Die gemessenen Daten der Stromstärke für die verschiedenen Winkel $\varphi$ des Polarisationsfilters .}
	\input{build/tabpolarisation.tex}
\end{table}
\begin{table}
	\centering
	\caption{Die gemessenen Daten der Stromstärke entlang der Horizontalen der $\text{TEM}_{\text{00}}$ Mode. Der Nullpunkt ist an die in Strahlrichtung linke Seite des Modenmusters gelegt.}
	\input{build/tabT00.tex}
\end{table}
\begin{table}
	\centering
	\caption{Die gemessenen Daten der Stromstärke entlang der Horizontalen der $\text{TEM}_{\text{01}}$ Mode. Der Nullpunkt ist an die in Strahlrichtung linke Seite des Modenmusters gelegt.}
	\input{build/tabT011.tex}
\end{table}
\begin{table}
	\centering
	\caption{Die gemessenen Daten der Stromstärke entlang der Horizontalen der $\text{TEM}_{\text{01}}$ Mode. Der Nullpunkt ist an die in Strahlrichtung linke Seite des Modenmusters gelegt.}
	\input{build/tabT012.tex}
\end{table}
\begin{table}
	\centering
	\caption{Die gemessenen Positionen, an denen sich Beugungsmaxima ausgebildet haben. Der Nullpunkt ist an die in Strahlrichtung linke Seite des Beugungsmusters gelegt.}
	\input{build/tabwelle.tex}
\end{table}
\begin{table}
	\centering
	\caption{Die gemessenen Positionen, an denen sich Beugungsmaxima ausgebildet haben. Der Nullpunkt ist an den linken Rand des verwendeten Schirms gesetzt.}
	\input{build/tabwelleneu.tex}
\end{table}