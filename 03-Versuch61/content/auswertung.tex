\section{Auswertung}
\label{sec:Auswertung}
Die Graphen wurden sowohl mit Matplotlib \cite{matplotlib} als auch NumPy \cite{numpy} erstellt. Die
Fehlerrechnung wurde mithilfe von Uncertainties \cite{uncertainties} durchgeführt.

\subsection{Überprüfung der Stabilitätsbedingung des Lasers}
Zunächst wird das in der Theorie angesprochene Stabilitätskriterium untersucht. Das Kriterium für einen stabilen Strahl im Experiment waren die sichtbare Streuung des Strahles an kleinen Staubkörnern in der Luft und die Stabilität des Strahles gegenüber kleinen Erschütterungen.
Hierbei ergibt sich auf Basis von \ref{eq:stabil} und dem zugehörigen Graphen in Abb. \ref{fig:stabilitat} für einen Resonator bestehend aus einem flachen Spiegel und einem konkaven Spiegel mit einem Krümmungsradius von $\SI{140}{\centi\meter}$ eine theoretische maximale stabile Resonatorlänge $L$ von $\SI{140}{\centi\meter}$. Diese maximale Resonatorlänge konnte auch im Messaufbau nachgewiesen werden. Für einen Resonator bestehend aus zwei konkaven Spiegeln mit einem Krümmungsradius von $\SI{140}{\centi\meter}$ konnte die theoretische maximale Länge von $\SI{280}{\centi\meter}$ nach Abb. \ref{fig:stabilitat} aufgrund der Größe des Aufbaus nicht nachgeprüft werden, es konnte jedoch ein stabiler Strahl bei einer Länge von $\SI{194}{\centi\meter}$ erzeugt werden. Daher ist zumindest das unterschiedliche Stabilitätsverhalten beider Resonatortypen experimentell bestätigt worden.
%erklärung, warum stabilität einbricht


\begin{figure}
	\centering
	\includegraphics[width=\linewidth-100pt,height=\textheight-100pt,keepaspectratio]{build/stabilitat.pdf}
	\caption{Die Graphen des Stabilitätskriterium für die zwei untersuchten Resonatorkombinationen in Abhängigkeit der Resonatorlänge $L$.}
	\label{fig:stabilitat}
\end{figure}




\subsection{Vermessung der $\text{TEM}_{00}$ Mode}
Nun folgt die Vermessung der TEM Moden, welche sich im Resonator ausbilden. Der Resonator bestand in dieser und den nachfolgenden Messungen aus zwei konkaven Spiegeln mit $R_i = \SI{140}{\centi\meter}$.
Für die $\text{TEM}_\text{00}$ Mode wurden die Daten in  Tabelle \ref{tab:T00} gemessen. Der resultierende Fit der $\text{TEM}_\text{00}$ Mode in Abb. \ref{fig:T00} wird mit einer nicht linearen Ausgleichsrechnung der Form

	$\frac{I}{I_\text{min}} = a \exp \left( \frac{-2*(x-c)^2}{b^2}\right)$
bestimmt. Dieser Fit wird aufgrund von Formel \eqref{eq:gaus} gewählt. Hierzu wird die Funktion curve\_fit aus der Numpy-Bibliothek Scipy \cite{scipy} gewählt. Für die Parameter des Fits ergibt sich:
\begin{gather*}
a = \num{1.45(4)e3}\\
b = \num{3.5(1)}\\
c = \num{5.28(6)}
\end{gather*}
Anhand von Abb. \ref{fig:T00} ist zu erkennen, dass die Form der $\text{TEM}_\text{00}$ Mode gut durch Formel \eqref{eq:gaus} dargestellt werden kann. Die Abweichungen der experimentellen Daten bezüglich des Fittes müssen in der Diskussion geklärt werden.


%a*np.exp(-2*((x-c)**2)/(b**2))

\begin{figure}
	\centering
	\includegraphics[width=\linewidth-100pt,height=\textheight-100pt,keepaspectratio]{build/T00.pdf}
	\caption{Der gemessenen Stromstärken im Verhältnis zur kleinsten gemessenen Stromstärke der $\text{TEM}_{00}$ Mode entlang der Horizontalen $x$ der Mode.Der Nullpunkt ist an die in Strahlrichtung linke Seite des Modenmusters gelegt.}
	\label{fig:T00}
\end{figure}

\begin{table}
	\centering
	\caption{Die gemessenen Daten der Stromstärke entlang der Horizontalen der $\text{TEM}_{\text{00}}$ Mode. Der Nullpunkt ist an die in Strahlrichtung linke Seite des Modenmusters gelegt.}
	\input{build/tabT00.tex}
	\label{tab:T00}
\end{table}


\subsection{Vermessung der $\text{TEM}_{01}$ Mode}
Nun wird zusätzlich auch die Form der $\text{TEM}_\text{01}$ Mode betrachtet. Für diese wurden die Werte in Tabelle \ref{tab:T01} gemessen. 
Der resultierende Fit der $\text{TEM}_\text{01}$ Mode in Abb. \ref{fig:T01} wird mit einer nicht linearen Ausgleichsrechnung der Form $\frac{I}{I_\text{min}} =  (x-c)^2 a \exp \left( \frac{-2*(x-c)^2}{b^2}\right)$. Dieser Fit wird aufgrund von Formel \eqref{eq:tem} gewählt. Auch hier wird die Funktion curve\_fit aus der Numpy-Bibliothek Scipy \cite{scipy} gewählt. Für die Parameter des Fits ergibt sich:
\begin{gather*}
a = \num{17.(1)}\\
b = \num{3.78(8)}\\
c = \num{6.21(6)}
\end{gather*}
Anhand von Abb. \ref{fig:T01} ist zu erkennen, dass die Form der $\text{TEM}_\text{01}$ Mode gut als entsprechender Fall der Formel \eqref{eq:tem} dargestellt werden kann. Die Abweichungen der experimentellen Daten bezüglich des Fittes müssen in der Diskussion geklärt werden.

\begin{figure}
	\centering
	\includegraphics[width=\linewidth-100pt,height=\textheight-100pt,keepaspectratio]{build/T01.pdf}
	\caption{Der gemessenen Stromstärken im Verhältnis zur kleinsten gemessenen Stromstärke der $\text{TEM}_{01}$ Mode entlang der Horizontalen $x$ der Mode.Der Nullpunkt ist an die in Strahlrichtung linke Seite des Modenmusters gelegt.}
	\label{fig:T01}
\end{figure}

\begin{table}
	\centering
	\caption{Die gemessenen Daten der Stromstärke entlang der Horizontalen der $\text{TEM}_{\text{01}}$ Mode. Der Nullpunkt ist an die in Strahlrichtung linke Seite des Modenmusters gelegt.}
	\input{build/tabT011.tex}
	\input{build/tabT012.tex}
	\label{tab:T01}
\end{table}

\subsection{Vermessung der Polarisation des Laserstrahles}
Nach der Vermessung der Moden wurde zusätzlich noch die Polarisation des entstandenen Laserstrahles untersucht. Die Messung ergab die Werte in Tabelle \ref{tab:polarisation}. Auf Basis dieser Werte und einem nicht linearen Fit der Form $I/I_\text{min} = a  \cos(b x + c)^2$
 aufgrund von Formel \eqref{eq:polar} ergibt sich der Graph in Abb. \ref{fig:polarisation}. Auch für diesen Fit wurde Scipy \cite{scipy} verwendet. Der Fit ergibt die folgenden Parameter:
 \begin{gather*}
	a = \num{2.36(6)e2}\\
	b = \num{1.00(1)}\\
	c = \num{1.20(4)}
	\end{gather*}
Es ist zu erkennen, dass der experimentelle Verlauf die theoretisch überlegte Form besitzt. Die Abweichungen der experimentellen Daten bezüglich des Fittes müssen in der Diskussion geklärt werden.
%was hat resonatorgestalt mit polarisation zu tun erklären
\begin{figure}
	\centering
	\includegraphics[width=\linewidth-100pt,height=\textheight-100pt,keepaspectratio]{build/Polarisation.pdf}
	\caption{Der gemessenen Stromstärken im Verhältnis zur kleinsten gemessenen Stromstärke in Abhängigkeit des Winkels des Polarisationsfilters.}
	\label{fig:polarisation}
\end{figure}


\begin{table}
	\centering
	\caption{Die gemessenen Daten der Stromstärke für die verschiedenen Winkel $\varphi$ des Polarisationsfilters .}
	\input{build/tabpolarisation.tex}
	\label{tab:polarisation}
\end{table}


\subsection{Bestimmung der ausgesendeten Wellenlänge des HeNe Lasers}
Zuletzt wurde die Wellenlänge des erzeugten Laserstrahles mithilfe eines Beugungsbildes gemessen. Unter Verwendung eines Gitters mit Gitterkonstanten $g = \SI{80}{\per\milli\meter}$ und einem Abstand vom Gitter zur Diode $b$ von $\SI{55}{\milli\meter}$ ergaben sich die Positionen der Intensitätsmaxima in Tabelle \ref{tab:tabwelle}.


Eine zweite Messreihe mit einem Gitter-Dioden Abstand von $\SI{440}{\milli\meter}$ ergab die Positionen in Tabelle \ref{tab:tabwelleneu}.





%\begin{table}
%	\centering
%	\caption{Die gemessenen Daten der Stromstärke entlang der Horizontalen der $\text{TEM}_{\text{01}}$ Mode. Der Nullpunkt ist an die in Strahlrichtung linke Seite des Modenmusters gelegt.}
	
%\end{table}
\begin{table}
	\centering
	\caption{Die gemessenen Positionen, an denen sich Beugungsmaxima ausgebildet haben. Der Nullpunkt ist an die in Strahlrichtung linke Seite des Beugungsmusters gelegt.}
	\input{build/tabwelle.tex}
	\label{tab:tabwelle}
\end{table}
\begin{table}
	\centering
	\caption{Die gemessenen Positionen, an denen sich Beugungsmaxima ausgebildet haben. Der Nullpunkt ist an den linken Rand des verwendeten Schirms gesetzt.}
	\input{build/tabwelleneu.tex}
	\label{tab:tabwelleneu}
\end{table}