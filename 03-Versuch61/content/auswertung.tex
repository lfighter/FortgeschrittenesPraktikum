\section{Auswertung}
\label{sec:Auswertung}
\subsection{Überprüfung der Stabilitätsbedingung des Lasers}




\begin{figure}
	\centering
	\includegraphics[width=\linewidth-100pt,height=\textheight-100pt,keepaspectratio]{build/stabilitat.pdf}
	\caption{Die Graphen des Stabilitätskriterium für die zwei untersuchten Resonatorkombinationen in Abhängigkeit der Resonatorlänge $L$ .}
	\label{fig:stabilitat}
\end{figure}




\subsection{Vermessung der T00 Mode}


\begin{figure}
	\centering
	\includegraphics[width=\linewidth-100pt,height=\textheight-100pt,keepaspectratio]{build/T00.pdf}
	\caption{Der gemessenen Stromstärken im Verhältnis zur kleinsten gemessenen Stromstärke der $\text{TEM}_{00}$ Mode entlang der Horizontalen $x$ der Mode.Der Nullpunkt ist an die in Strahlrichtung linke Seite des Modenmusters gelegt.}
	\label{fig:T00}
\end{figure}



\subsection{Vermessung der T01 Mode}



\begin{figure}
	\centering
	\includegraphics[width=\linewidth-100pt,height=\textheight-100pt,keepaspectratio]{build/T01.pdf}
	\caption{Der gemessenen Stromstärken im Verhältnis zur kleinsten gemessenen Stromstärke der $\text{TEM}_{01}$ Mode entlang der Horizontalen $x$ der Mode.Der Nullpunkt ist an die in Strahlrichtung linke Seite des Modenmusters gelegt.}
	\label{fig:T01}
\end{figure}



\subsection{Vermessung der Polarisation des Laserstrahles}


\begin{figure}
	\centering
	\includegraphics[width=\linewidth-100pt,height=\textheight-100pt,keepaspectratio]{build/Polarisation.pdf}
	\caption{Der gemessenen Stromstärken im Verhältnis zur kleinsten gemessenen Stromstärke in Abhängigkeit des Winkels des Polarisationsfilters.}
	\label{fig:polarisation}
\end{figure}




\subsection{Bestimmung der ausgesendeten Wellenlänge des HeNe Lasers}




\begin{table}
	\centering
	\caption{Die gemessenen Daten der Stromstärke für die verschiedenen Winkel $\varphi$ des Polarisationsfilters .}
	\input{build/tabpolarisation.tex}
\end{table}
\begin{table}
	\centering
	\caption{Die gemessenen Daten der Stromstärke entlang der Horizontalen der $\text{TEM}_{\text{00}}$ Mode. Der Nullpunkt ist an die in Strahlrichtung linke Seite des Modenmusters gelegt.}
	\input{build/tabT00.tex}
\end{table}
\begin{table}
	\centering
	\caption{Die gemessenen Daten der Stromstärke entlang der Horizontalen der $\text{TEM}_{\text{01}}$ Mode. Der Nullpunkt ist an die in Strahlrichtung linke Seite des Modenmusters gelegt.}
	\input{build/tabT011.tex}
\end{table}
\begin{table}
	\centering
	\caption{Die gemessenen Daten der Stromstärke entlang der Horizontalen der $\text{TEM}_{\text{01}}$ Mode. Der Nullpunkt ist an die in Strahlrichtung linke Seite des Modenmusters gelegt.}
	\input{build/tabT012.tex}
\end{table}
\begin{table}
	\centering
	\caption{Die gemessenen Positionen, an denen sich Beugungsmaxima ausgebildet haben. Der Nullpunkt ist an die in Strahlrichtung linke Seite des Beugungsmusters gelegt.}
	\input{build/tabwelle.tex}
\end{table}
\begin{table}
	\centering
	\caption{Die gemessenen Positionen, an denen sich Beugungsmaxima ausgebildet haben. Der Nullpunkt ist an den linken Rand des verwendeten Schirms gesetzt.}
	\input{build/tabwelleneu.tex}
\end{table}