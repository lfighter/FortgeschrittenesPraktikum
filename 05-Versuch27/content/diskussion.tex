
\section{Diskussion}
\label{sec:Diskussion}
Die Auswertung hat einige Ergebnisse erbracht, welche nun noch zu diskutieren sind. Die bestimmten Parameter der B-Feld Kalibrierung zeigen nur geringe Unsicherheiten. Es ist jedoch zu vermuten, dass die bestimmten B-Felder insgesamt zu niedrig sind. Grund hierfür ist wahrscheinlich, dass die verwendete Hall-Sonde zu tief in das B-Feld gehalten wurde, sodass die Sonde nicht die maximale B-Feld Stärke messen konnte. %nochmal auf die form des b feldes eingehen
Diese Vermutung wird durch die experimentell bestimmten Landefaktoren verstärkt. Diese fallen insgesamt zu groß aus. So ist der bestimmte Landefaktor des normalen Zeemanneffektes mit $\num1.17(1) $ ca. 17\% größer als der theoretisch bestimmte. Die Unsicherheiten sind jedoch nur gering. Dieses Schema setzt sich bei den Ergebnissen des anormalen Zeemanneffektes fort.  Der gemessene Landefaktor des $\sigma$ Übergang ist 12\% zu groß, der des $\pi$ Übergangs sogar 31\%. Daher zeigen die Befunde nur Ähnlichkeiten mit den erwarteten Werten, für eine Bestätigung der Theorie sind sie hingegen ungeeignet.