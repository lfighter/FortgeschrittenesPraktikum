\section{Auswertung}
\label{sec:Auswertung}
Die Graphen wurden sowohl mit Matplotlib \cite{matplotlib} als auch NumPy \cite{numpy} erstellt. Die
Fehlerrechnung wurde mithilfe von Uncertainties \cite{uncertainties} durchgeführt. Die Vermessung der Photographien wurde mit Gimp durchgeführt. Um die einzelnen Linien der Spektren besser zu erkennen wurden Kontrast, Helligkeit, Sättigung und Farbtiefe der Photographien für die Auswertung modifiziert.%cite gimp angeben

\subsection{Kalibrierung des angelegten B-Feldes}
Zunächst wird die Stärke des B-Feld des Helmholtzspulenpaares, in dessen Zentrum sich die Cadmiumlampe befindet, in Abhängigkeit der angelegten Stromstärke vermessen. Es ergaben sich die Werte in Tabelle \ref{tab:Bfeld}. Mithilfe einer lineare Ausgleichsrechnung der Form $a*x+b$ ergibt sich der Graph in Abb. \ref{fig:BvonI}. Für die beiden Konstanten $a$ und $b$ folgt: 
\begin{gather}
	a = \SI{}{\tesla\per\ampere}%werte einfügen
	b = \SI{}{\tesla}
\end{gather}

\subsection{Vermessung des normalen Zeemanneffekts an der roten $\sigma$ Cadmiumlinie }%vll besserer titel
Nun wird der Landefaktor der roten $\sigma$ cadmiumlinie bestimmt. Hierzu wurden die Photographien des Interferenzmusters ohne B-Feld in Abb. \ref{fig: 1} und des mit B-Feld in Abb. \ref{fig: 2} aufgenommen. Anschließend wurden die Abstände der Ordnungen aus Abb. \ref{fig: 1} und die Abstände der Aufspaltungen innerhalb der Ordnungen in \ref{fig: 2} vermessen. Es ergeben sich die Werte in  den Tabellen \ref{} und \ref{}.

Auf Basis von Formel \eqref{eq:} ergeben sich die einzelnen $\Delta\lambda$ in Tabelle \ref{tab:}. Ein mittleres $\Delta\lambda$ wird über eine konstante Ausgleichsrechnung der Form $c$ zu $\SI$ bestimmt. Der Landefaktor berechnet mithilfe von Formel \eqref{} und den Konstanten %konstanten angeben
zu :::::::.