
\section{Durchführung}
\label{sec:Durchführung}
Es wird der experimentelle Aufbau, wie in Abschnitt \ref{sec:Aufbau} beschrieben, montiert. Der Elektromagnet wird kalibriert, indem mit Hilfe einer Hall-Sonde das Magnetfeld an der Cadmium-Lampe in Abhängigkeit von der Stromstärke gemessen wird. Für den normalen Zeeman-Effekt werden zwei Bilder mit der Digitalkamera aufgenommen. Das eine Bild wird ohne eingeschalteten Elektromagneten, das andere Bild mit eingeschalteten Elektromagneten aufgenommen. Es muss dabei darauf geachtet werden, dass der Aufbau zwischen den Aufnahmen nicht verändert werden darf. Weiterhin muss bei der Aufnahme mit Magnetfeld die $\pi$-Linie mit Hilfe des Polarisationsfilters herausgefiltert werden. Auch sollte das Magnetfeld so gewählt werden, dass die Aufspaltung gut zu erkennen ist. Für den anomalen Zeeman-Effekt werden drei Bilder mit der Digitalkamera aufgenommen. Zwei Bilder werden wie beim normalen Zeeman-Effekt aufgenommen. Zusätzlich wird noch ein Bild bei hoher Magnetfeldstärke von der $\pi$-Linie aufgenommen. Es darf der Aufbau zwischen den Aufnahmen wieder nicht verändert werden. Es muss darauf geachtet werden, dass der Polarisationsfilter nicht die $\pi$-Linie herausgefiltert.
