
\section{Durchführung}
\label{sec:Durchführung}
Es wird der Laser wie im Aufbau \ref{sec:Aufbau} beschrieben montiert. Dazu werden zunächst eine Lochblende mit Fadenkreuz direkt vor dem Justierlaser und möglichst weit von diesem entfernt auf der Schiene befestigt. Es wird der Justierlaser angeschaltet. Der Laser sollte nun durch beide Blenden hindurch strahlen. Als nächstes wird ein Spiegel zwischen diesen auf der Schiene montiert. Dieser wird mit den Justierschrauben so eingestellt, dass der Reflex des Justierlasers wieder genau in die Mitte des Fadenkreuzes fällt. Nun wird das Laserrohr hinter dem Spiegel angebracht. Mit den Justierschrauben wird das Rohr nun so bewegt, dass der Laser möglichst mittig durch dieses hindurch fällt, so dass der Laser hinter diesem weiterhin geradlinig verläuft. Nun wird der zweite Spiegel des Resonators hinter dem Laserrohr montiert. Es sollte dabei darauf geachtet werden, dass die Resonatorlänge die Stabilitätsbedingung \eqref{eq:stabil} erfüllt. Der Spiegel sollte zur einfacheren Justierung in einem möglichst großen Abstand zum Laserrohr montiert werden. Er wird nun so gedreht, dass der reflektierte Strahl sich mit dem Strahl vor der Reflexion überlappt. Dies kann mit durchscheinenden Papier überprüft werden, indem dieses in den Strahl gehalten wird. Nun wird der Justierlaser ausgeschaltet und die Spannungsquelle, die mit den Elektroden des Laserrohres verbunden ist wird angeschaltet. Das Laserrohr sollte beginnen orange zu leuchten. Meistens sind die Spigel noch nicht gut genug justiert, damit sich ein Laserstrahl ausbildet. Es wird deshalb vorsichtig an den Justierschrauben der Spiegel gedreht und gehofft, dass sich eine Lasertätigkeit einstellt. Ist dies nicht der Fall wird der Justiervorgang wiederholt. 

Zum Überprüfen der theoretischen Stabilitätsbedingung wird der Resonatorabstand beim laufenden Laser vergrößert, indem die Resonatorspiegel verschoben werden. Dabei muss häufiger nachjustiert werden, damit der Laserstrahl erhalten bleibt. Der maximale Resonatorabstand, bei dem der Laserstrahl stabil bleibt wird notiert. Dies wird für zwei verschiedene Resonatoren durchgeführt.

Zum Vermessen der $\text{TEM}_{mnq}$-Moden wird die Resonatorlänge möglichst klein gehalten. Dazu werden die Spiegel wie zuvor verschoben. Zum Stabilisieren der Moden wird nun ein dünner Draht zwischen das Laserrohr und dem Resonatorspiegel in den Laserstrahl gebracht. Der Draht kann verschoben werden, um verschiedene Moden zu stabilisieren. Hinter dem Resonatorspiegel kann eine Linse befestigt werden, um das Bild der Moden zu vergrößern. Mit einer Photodiode, die hinter der Linse montiert wird, sollen mindestens zwei $\text{TEM}_{mnq}$-Moden vermessen werden.

Zur Bestimmung der Polarisation des Lasers wird ein Polarisationsfilter hinter dem zweiten Resonatorspiegel montiert. Hinter diesem wird die Photodiode so montiert, dass der Laser besonders gut in die Photodiode fällt. Nun wird der Polarisationsfilter in mehreren Schritten um $\SI{360}{\degree}$ gedreht. Für jeden Schritt wird die Intensität notiert.

Um die Wellenlänge des Lasers mithilfe der Beugung an einem Gitter zu bestimmen, wird ein Gitter hinter dem zweitem Resonatorspiegel montiert und hinter diesem eine Photodiode. Der Abstand zwischen Gitter und Photodiode wird notiert, sowie die Positionen einiger Maxima.